\subsection{Allgemeine Methodik / Vorgehen / Literaturüberblick}\label{subsec:Methodik}
Bspw.: Business Model Canvas / CRSIP-DM-Cycle / Behavior Driven Development / SCRUM)
Design Science: Rigor-Cycle

\subsection{Designphase}\label{subsec:Designphase}
Datenmodell / Frontend / Backend
Design Science: Design-Cycle

\subsubsection{Backend}
Der erste Schritt bei der Backend entwicklung war es, einen geeigneten Startpunkt zu finden.
Hierbei wurde entschieden, dass zuerst eine Implementierung der Datenbankverbindung erfolgen sollte, da diese als Grundlage für die weitere Entwicklung dient.
Dies wurde mit der Bibliothek mysql2 realisiert, welche eine erweiterte Funktionalität gegenüber der Standardbibliothek bietet und diverse Probleme behebt.
Diese implementierung war erst möglich, nachdem die Grundstruktur der Datenbank aufgesetzt wurde.
Einen Einblick in den Code gibt Listing\ref{lst:dbconnector}.

% \lstinputlisting[language=JavaScript,label={lst:dbconnector}]{../../server/dbConnections/connectToMYSQL.js}

Als Nächstes wurde sich dazu entschieden, ein einfaches Sign Up und Login System zu programmieren.
Hierbei wurde auf die Bibliothek bcrypt zurückgegriffen, welche das Hashen von Passwörtern erleichtert.
Die SQL queries, die hier genutzt werden, wurden ebenfalls mit der mysql2 Bibliothek realisiert.
Die einzigen Nutzerdaten, die hierbei gespeichert werden, sind der Username, die E-Mail und das Passwort.
Beim Login wurde darauf geachtet, das Nutzer Username und E-Mail benutzen können, um sich einzuloggen.

Nun stellte sich die Frage, wie es möglich ist, dass Nutzer nur auf ihre eigenen Sammlungen zugreifen können.
Beim Recherchieren sind wir auf das Konzept von Sessions gestoßen und wollten diese ausprobieren.

Parallel wurde eine Funktion entwickelt, die Daten aus der Datenbank anhand des Tabellennamens ausliest.
Diese werden dann in eine Form gebracht, welche das Frontend nutzen kann, um Beispieldaten in einer Tabelle einzupflegen.



\subsubsection{Datenbank}

Bei der Einrichtung der Datenbank stellt sich als erste Hürde der Umgang mit Docker Container heraus.
Zwar waren grundlegende Kentnisse über Docker vorhanden, doch eine Datenbank praktisch in einem Container hochzufahren und diese dann mit dem Code und der Programmierumgebung zu verknüpfen, war eine neue Herausforderung.
Hierzu mehr in Kapitel~\ref{subsec:Herausforderungen}.
Wie in der Planung entschieden, sollen zwei Datenbanken genutzt werden - eine MySQL Datenbank für strukturierte Daten und eine MongoDB Datenbank für unstrukturierte Daten.
Im ersten Schritt wurde die MySQL Datenbank aufgesetzt und mit Tabellen für Nutzerdaten und den drei Pre-Sets für Sammlungen befüllt.
Die Nutzerdaten beinhalten Username, E-Mail und Passwort.
Für die Themenbereiche Video Spiele und Parfum wurden jeweils eine Tabelle erstellt, die für das Thema passende Spalten beinhalten.

\subsection{Herausforderungen}\label{subsec:Herausforderungen}

Bei der Programmierung des Projekts gab es einige Stellen, an denen das Team auf Probleme, und vor allem auf steile Lernkurven stieß.
Eine dieser Stellen war der Umgang mit Docker Containern.
Ein gewisses Grundverständnis war vorhanden, doch praktische Erfahrung war im gesamten Team nicht vorhanden.
Zwar was das Hochfahren einer Datenbank in einem Container schnell erreicht, doch ein Verständnis für Datenpersistenz und Volumes zu entwickeln, benötigte seine Zeit.
Auch im späteren Verlauf des Projekts, als es darum ging alle Applikationskomponenten in einer Docker-Compose Datei zusammenzuführen, stellte sich als Herausforderung heraus.
Wie sich Dockerfiles in dem ganzen System einordnen und wo sie genutzt werden, war ebenfalls neu für das Team.
Allgemein hat das Einfinden in Docker dem Team mehr Zeit als erwartet abverlangt.

Eine weitere Herausforderung war das Einfinden in die verschiedenen Bibliotheken, die bei Webentwicklung mit JavaScript genutzt werden.
Zwar wurde in der Planung bereits einige Bibliotheken festgelegt, die genutzt werden sollten, doch während der Entwicklung kamen einige neue hinzu.
Ein Beispiel hierfür ist die Bibliothek express-session, die für das Session-Management genutzt wird.
Da zur Planung noch kein detailliertes Verständnis darüber vorhanden war, welche Komponenten bei der Entwicklung einer solchen Anwendung benötigt werden, wurden Thematiken wie Session Management erst während der Entwicklung entdeckt.
Hierdurch kam es an einigen Stellen zu steilen, jedoch unvermeidbaren Lernkurven.

Im nächsten Schritt wurde die MongoDB implementiert.
Dies ist das erste Mal, dass das Team mit einer Dokumenten orientierten Datenbank arbeitet, daher musste erstmal ein Verständnis über den Aufbau einer solchen Datenbank geschaffen werden.
Zunächst wurde versucht vergleiche mit einer SQL Datenbank herzustellen, wobei schnell auffiel, dass Konzepte wie Schemata hier Collections sind.
Das integrieren der Datenbank in der Programmierumgebung war schnell erledigt, da dies analog zu der MySQL Datenbankverbindung erfolgt ist.
Das auslesen und schreiben von Daten in die Datenbank war dank der MongoDB Bibliothek einfach erledigt.
Funktionen für die Verbindung mit der Datenbank wurden in einer separaten Datei ausgelagert.
\subsection{Prototypvorstellung}\label{subsec:Prototyp}
Design Science: Artefakt