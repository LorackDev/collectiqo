\subsection{Motivation}\label{subsec:Motivation}

Das Sammeln von Kulturgütern ist ein Bedürfnis von vielen verschiedenen Menschen und eine faszinierende Praxis, die tiefe Einblicke in Geschichte, Identität und ästhetische Vorlieben gewährt.
Es ist eine Leidenschaft, die Menschen verschiedener Hintergründe und Interessen verbindet, sei es das Sammeln von Parfumflaschen, Videospielen oder historischen Münzen.
Diese vielfältigen Sammlungen zeugen von der menschlichen Neugierde und dem Bedürfnis, materielle Kulturgüter zu bewahren und zu schätzen. \par
Trotz des reichen Angebots an Online-Plattformen zur Verwaltung einzelner Sammlungen fehlt allerdings bisher eine universelle Plattform, die es ermöglicht, mehrere Sammlungen unterschiedlicher Güter in einem Zug zu digitalisieren. \par
Das Praxisprojekt namens Collectiqo strebt danach, diese Lücke zu füllen, indem sie eine ansprechende und zugängliche Plattform bietet, um diverse Sammlungen zu organisieren, zu dokumentieren und zu präsentieren, unabhängig von deren thematischer Ausrichtung.

\subsection{Marktanalyse / Technologische Grundlagen}\label{subsec:Marktanalyse-TechnologischeGrundlagen}
Zielgruppe, Marktabgrenzung, theoretische Grundlagen, Literaturüberblick\linebreak
Design Science: Relevance-Cycle