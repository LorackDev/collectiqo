\subsection{Motivation}\label{subsec:Motivation}


Das Sammeln von Kulturgütern ist ein Bedürfnis von vielen verschiedenen Menschen und eine Praxis, die Einblicke in Geschichte, Identität und ästhetische Vorlieben gewährt.
Es ist eine Leidenschaft, die Menschen verschiedener Hintergründe und Interessen verbindet, sei es das Sammeln von Parfumflaschen, Videospielen oder historischen Münzen.
Das Projektteam dient hierbei als Beispiel, da jedes Mitglied mindestens eine Sammlerleidenschaft verfolgt.
Bei mehreren Sammlungen, die man pflegt, fällt jedoch ein erhöhter Pflegeaufwand auf. \par
Trotz des reichen Angebots an Online-Plattformen zur Verwaltung einzelner Sammlungen fehlt bisher eine universelle Plattform, die es ermöglicht, mehrere Sammlungen unterschiedlicher Güter einheitlich zu digitalisieren. \par
Daher ist es das Ziel dieses Praxisprojektes, eine entsprechende Webseite aufzubauen, welche für Sammlungen jeglicher Art sowohl vorgefertigte als auch anpassbare Vorlagen bietet, mit denen diese organisiert, dokumentiert und auch präsentiert werden können.
Dies würde den Pflegeaufwand für Sammler verringern und spricht im großen Interesse des Projektteams.


\subsection{Marktanalyse / Technologische Grundlagen}\label{subsec:Marktanalyse-TechnologischeGrundlagen}

Um Anforderungen an das Projekt zu definieren und die Plattform bestmöglich zu differenzieren, wurde zwecks diesem eine Marktanalyse zu Sammlerplattformen betrieben.
Hierbei wurden auf Aspekte wie die Zielgruppe, das Angebot auf dem Markt und einzelner Plattformen, die Benutzerfreundlichkeit und aktuelle Markttrends geachtet. \par
Die Zielgruppe bezieht sich, wie in KAPITEL 1-1 beschrieben, auf Sammler verschiedener Sachgüter. \linebreak


Das Angebot der Plattformen unterscheidet sich in ihrer Spezialität und ihrer Zielrichtung. Zwei der ausgewerteten Webseiten bieten für unterschiedliche Sammelgüter die Dokumentierung, Präsentation und Kapitalisierung von Sammelgütern an.
Restliche Webseiten bieten speziell für eine Kategorie von Sammlerobjekten dieselben Funktionen.
Keine Webseite bietet die Eigenerstellung einer Sammlung beziehungsweise Kategorie an. \par

Die Benutzerfreundlichkeit ist ein wichtiger Faktor für den Erfolg von Sammlerplattformen.
Eine intuitive Benutzeroberfläche, ein ansprechendes UI-Design, klare Kategorisierung der Sammelgüter und einfache Such- und Filterfunktionen tragen dazu bei, dass Benutzer gerne die Plattform benutzen.
Hierbei waren etwa die Hälfte der untersuchten Webseiten subjektiv gut gestaltet und erfüllen die genannten Anforderungen, die restlichen Seiten schienen im Design veraltet oder unhandhablich. \par

In den letzten Jahren hat die Nachfrage nach Sammlerobjekten zugenommen, sowohl von traditionellen Sammlern als auch von neuen Zielgruppen, die das Potenzial von Sammlerstücken als Investition erkannt haben.
Die Popularität von Online-Marktplätzen, wie beispielsweise eBay, hat ebenfalls dazu beigetragen, den Sammlermarkt zu beleben und den Zugang zu seltenen und einzigartigen Stücken zu erleichtern. \par
Insgesamt ist der Markt für Sammlerplattformen lebendig und vielfältig, jedoch nicht individualisierbar für Nischensammlungen. \par
Da die Analyse anhand der vorhandenen Daten verschiedener Plattformen betrieben wurde und keine Quellen bezüglich Personenbefragungen gefunden wurden, wurde eine anonyme Umfrage erstellt, in der Personen befragt wurden, ob sie Sammeln, wie Sie ihre Kollektion handhaben, und ob diese Interesse an einer benutzerfreundlichen Universalplattform zeigen. \linebreak

(Umfrageergebnisse \& für Analyse Quellen und Grafiken incoming)

