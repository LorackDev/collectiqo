\section{Herausforderungen}\label{sec:Herausforderungen}
Bei der Programmierung des Projekts gab es einige Stellen, an denen das Team auf Probleme, und vor allem auf steile Lernkurven stieß.

\subsection{Frontend}\label{subsec:Frontend}
Hier Herausforderungen bei der Frontendentwicklung einfügen

\subsection{Backend}\label{subsec:Backend}
Eine dieser Stellen war der Umgang mit Docker Containern.
Ein gewisses Grundverständnis war vorhanden, doch praktische Erfahrung war im gesamten Team nicht vorhanden.
Zwar was das Hochfahren einer Datenbank in einem Container schnell erreicht, doch ein Verständnis für Datenpersistenz und Volumes zu entwickeln, benötigte seine Zeit.
Auch im späteren Verlauf des Projekts, als es darum ging alle Applikationskomponenten in einer Docker-Compose Datei zusammenzuführen, stellte sich als Herausforderung heraus.
Wie sich Dockerfiles in dem ganzen System einordnen und wo sie genutzt werden, war ebenfalls neu für das Team.
Allgemein hat das Einfinden in Docker dem Team mehr Zeit als erwartet abverlangt.

Eine weitere Herausforderung war das Einfinden in die verschiedenen Bibliotheken, die bei Webentwicklung mit JavaScript genutzt werden.
Zwar wurde in der Planung bereits einige Bibliotheken festgelegt, die genutzt werden sollten, doch während der Entwicklung kamen einige neue hinzu.
Ein Beispiel hierfür ist die Bibliothek express-session, die für das Session-Management genutzt wird.
Da zur Planung noch kein detailliertes Verständnis darüber vorhanden war, welche Komponenten bei der Entwicklung einer solchen Anwendung benötigt werden, wurden Thematiken wie Session Management erst während der Entwicklung entdeckt.
Hierdurch kam es an einigen Stellen zu steilen, jedoch unvermeidbaren Lernkurven, die jedoch auch entsprechenden zeitlichen Aufwand mit sich gebracht haben.

Da das Team zuvor noch nie eine Webanwendung entwickelt hat, fehlte es an Verständnis, wie viel Komponenten Teil einer solchen Anwendung sind.
So kam es, dass während der Entwicklung immer wieder neue Komponenten entdeckt wurden, die in der Planung nicht berücksichtigt wurden.
Ein konkretes Beispiel hierfür ist das Session-Management.

\subsection{Datenbanksystem}\label{subsec:Datenbanksystem}
Die Herausforderungen, die beim Einrichten des Datenbanksystems entstanden sind, teilen sich in zwei Punkte auf.
Der erste Punkt hängt damit zusammen, dass das Einrichten der Datenbanken gleichzeitig die erste praktische Nutzung von Docker Containern war.
