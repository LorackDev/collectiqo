\section{Rückblick / Retrospektive}\label{sec:rueckblick-retrospektive}

Dieser Abschnitt gibt einen Vergleich zwischen den geplanten Zielen und den tatsächlich erreichten Zielen.
Daraufhin wird geschaut, welche Verbesserungen in der Durchführung nächstes Semester umgesetzt werden könnten.
Schlussendlich wird ein Ausblick auf zukünftige Ziele und Funktionen gegeben, wovon einige aus den geplanten Zielen für dieses Semester übernommen werden und einige neue hinzukommen, die dem Team bei der Umsetzung eingefallen sind.

\subsection{Abweichungen der geplanten Ziele}\label{subsec:abweichungen-der-geplanten-ziele}
Während der Umsetzung wurden immer wieder die Definitions of done in Betracht gezogen.
Dieses Kapitel betrachtet die einzelnen Ziele und schaut, ob diese erreicht wurden und was der aktuelle Stand in der Umsetzung ist.
Hierbei wurden die Ziele auf die beschränkt, die direkt etwas mit der Umsetzung des Projekts zu tun haben.
Ziele wie die erfolgreiche Abgabe des Projekts wurden somit ausenvorgelassen.

\begin{table}[h]
    \centering
    \begin{tabular}{|p{0.4\textwidth}|p{0.3\textwidth}|p{0.3\textwidth}|}
        \hline
        \textbf{Definition of done} & \textbf{Status} & \textbf{Kommentar} \\
        \hline
        Die Idee des Projektes ist erstellt &  Abgeschlossen & In der Planungsphase konnte das Team sich auf eine gemeinsame Vision für das Projekt einigen. \\
        \hline
        Auswahl der genutzten Software- bzw. Projektwerkzeuge ist erfolgt & Abgeschlossen & In der Planungsphase wurde eine übersicht über alle Tools erstellt, die während der Programmierung genutzt werden sollen. \\
        \hline
        Projektplanung ist angelegt & Abgeschlossen & Die Planung des Projekts wurde vollumfänglich angelegt in diesem Semester. \\
        \hline
        Anlegen der Projektumgebung mithilfe der Software- bzw. Projektwerkzeuge erfolgte & Erreicht  &  \\
        \hline
        Backend-System ist implementiert & Grundlagen abgeschlossen / In Bearbeitung & Viele Funktionen für die Authetifizierung, das Erstellen und Bearbeiten von Sammlungen sowie die Datenbankverbinden wurden bereits implementiert.
        Die Grundfunktionen wurden somit erfolgreich implementiert.
        Im folgenden Semester sollen die Backendfunktionen weiter ausgebaut werden. \\
        \hline
        Frontend-System ist implementiert & Grundlagen abgeschlossen / In Bearbeitung &  \\
        \hline
        Datenbanksystem ist implementiert & In Bearbeitung &  \\
        \hline
        Verknüpfung der drei Systeme erfolgte & In Bearbeitung &  \\
        \hline
        Account-Erstellung und Benutzer-Login ist möglich & Abgeschlossen &  \\
        \hline
        Vorhandene „Sammlungen“-Templates können genutzt werden & Abgeschlossen &  \\
        \hline
        Benutzer kann erfolgreich eigene Templates anlegen & In Bearbeitung & Die Backend Funktion wurde bereits implementiert, im Frontend fehlt jedoch ein Interface zum Erstellen eigener Vorlagen. \\
        \hline
        Benutzerrechte-Einstellungen sind passend & Verschoben & Erste Drafts von Nutzereinstellungen wurden bereits erstellt, die Umsetzung wurde jedoch zeitbedingt auf das nächste Semester verschoben. \\
        \hline
        Alle notwendigen Tests erfolgreich & Gescheitert / Verschoben & Das Erstellen von Integrationstests hat sich als zeitlich zu aufwendig erwiesen.
        Die Implementierung von komplexeren Tests wurde auf das nächste Semester verschoben. \\
        \hline
    \end{tabular}
    \caption{Your caption}
    \label{your-label}
\end{table}

\subsection{Retrospektive}\label{subsec:Retrospektive}

\subsection{Ausblick / Zukünftige Ziele und Funktionen}\label{subsec:ausblick-zukuenftige-ziele-und-funktionen}