\section{Rückblick / Retrospektive}\label{sec:rueckblick-retrospektive}

Dieser Abschnitt gibt einen Vergleich zwischen den geplanten Zielen und den tatsächlich erreichten Zielen.
Daraufhin wird geschaut, welche Verbesserungen in der Durchführung nächstes Semester umgesetzt werden könnten.
Schlussendlich wird ein Ausblick auf zukünftige Ziele und Funktionen gegeben, wovon einige aus den geplanten Zielen für dieses Semester übernommen werden und einige neue hinzukommen, die dem Team bei der Umsetzung eingefallen sind.

\subsection{Abweichungen der geplanten Ziele}\label{subsec:abweichungen-der-geplanten-ziele}
Während der Umsetzung wurden immer wieder die Definitions of done in Betracht gezogen.
Dieses Kapitel betrachtet die einzelnen Ziele und schaut, ob diese erreicht wurden und was der aktuelle Stand in der Umsetzung ist.
Hierbei wurden die Ziele auf die beschränkt, die direkt etwas mit der Umsetzung des Projekts zu tun haben.
Ziele wie die erfolgreiche Abgabe des Projekts wurden somit ausenvorgelassen.

\begin{table}[h]
    \centering
    \begin{tabular}{|p{0.2\textwidth}|p{0.2\textwidth}|p{0.6\textwidth}|}
        \hline
        \textbf{Definition of Done} & \textbf{Status} & \textbf{Kommentar} \\
        \hline
        Die Idee des Projektes ist erstellt &  Abgeschlossen & In der Planungsphase konnte das Team sich auf eine gemeinsame Vision für das Projekt einigen. \\
        \hline
        Auswahl der genutzten Software- bzw. Projektwerkzeuge ist erfolgt & Abgeschlossen & In der Planungsphase wurde eine übersicht über alle Tools erstellt, die während der Programmierung genutzt werden sollen. \\
        \hline
        Projektplanung ist angelegt & Abgeschlossen & Die Planung des Projekts wurde vollumfänglich angelegt in diesem Semester, wobei ergänzende Planungsschritte für künftige Semester möglich sind. \\
        \hline
        Anlegen der Projektumgebung mithilfe der Software- bzw. Projektwerkzeuge erfolgte & Erreicht  &  Bei allen Teammitgliedern wurde eine Entwicklungsumgebung eingerichtet, auf der alle Tools, die in der Planungphase festgelegt wurden, installiert bzw. lauffähig sind. \\
        \hline
        Backend-System ist implementiert & Grundlagen abgeschlossen & Viele Funktionen für die Authentifizierung, das Erstellen und Bearbeiten von Sammlungen sowie das Datenbankverbinden wurden bereits implementiert.
        Die Grundfunktionen wurden somit erfolgreich implementiert. \\
        \hline
        Frontend-System ist implementiert & Grundlagen abgeschlossen &  Grundlegende Frontend Seiten wurden implementiert, die zum Großteil den bestehenden Funktionsumfang des Backends abdecken.\\
        \hline
        Datenbanksystem ist implementiert & Abgeschlossen & Die Datenbanken wurden passend für den aktuellen Stand des Front- und Backends implementiert und decken alle aktuellen Anforderungen ab. \\
        \hline
        Verknüpfung der drei Systeme erfolgte & In Bearbeitung & Während die Datenbanken vollumfänglich mit dem Backend interagieren war es noch nicht möglich, alle Backend Funktionen mit dem Frontend zu verbinden.
        Auf beiden Seiten gibt es ein Overhead an Funktionen, die von der jeweils anderen Seite noch nicht abgedeckt werden. \\
        \hline
        Account-Erstellung und Benutzer-Login ist möglich & Abgeschlossen &  Über die Sign-Up Seite werden Nutzer erfolgreich in der MySQL Datenbank hinterlegt.
        Beim Anmelden über die Loginseite wird gegen kontrolliert, ob Nutzername und Passwort mit dem Eintrag in der Datenbank übereinstimmen.
        \\
        \hline
        Vorhandene „Sammlungen“-Templates können genutzt werden & Abgeschlossen & Das Frontend erlaubt die Auswahl eines Templates zum Erstellen einer Sammlung.
        Beim Speichern der Sammlung wird das Template aus der MongoDB extrahiert und genutzt um darauf basierend eine Sammlung in passende MongoDB Collection zu schreiben.
        \\
        \hline
        Benutzer kann erfolgreich eigene Templates anlegen & In Bearbeitung & Die Backend Funktion inklusive Datenbankverbindung wurde bereits implementiert, im Frontend fehlt jedoch ein Interface zum Erstellen eigener Vorlagen. \\
        \hline
        Benutzerrechte-Einstellungen sind passend & Verschoben & Erste Drafts von Nutzereinstellungen wurden bereits erstellt, die Umsetzung wurde jedoch zeitbedingt auf das nächste Semester verschoben. \\
        \hline
        Alle notwendigen Tests erfolgreich & Gescheitert / Verschoben & Das Erstellen von einfachen Unit-Tests war zwar erfolgreich, doch das Erstellen von Integrationstests hat sich als zeitlich zu aufwendig erwiesen.
        Die Implementierung von komplexeren Tests wurde auf das nächste Semester verschoben. \\
        \hline
    \end{tabular}
    \caption{Your caption}
    \label{your-label}
\end{table}

\subsection{Retrospektive}\label{subsec:Retrospektive}

Während der Durchführungsphase wurde schnell klar, dass die Lernkurve mit den gewählten Technologien steiler war als gedacht.
Dies lag vor allem daran, dass mehr Komponenten in die Entwicklung einer Website fielen als gedacht.
Hierbei hätte sich das Team während der Planungsphase intensiver damit auseinandersetzen sollen, welche Bausteine für eine Webanwendung benötigt werden.
Die Toolauswahl fiel hierbei im Programmierbereich oberflächlich statt, weshalb nur Komponenten wie die gewählte Programmiersprache angegeben wurden, jedoch keine spezifischen Bibliotheken.
An dieser Stelle hätte man kleinteiliger planen sollen, um die Ziele für dieses Semester realistischer setzen zu können.

Gleichzeitig stellte diese Herausforderung einen hohen Lerneffekt dar.
Dadruch, das fast ausschließlich Tools genutzt wurden, mit denen das Team noch keine Erfahrung hatte, konnten sich so viele Kenntnisse angeeignet werden.
Besonders sticht hierbei der Umgang mit Docker heraus.
Die Docker Kenntnisse, die sich durch dieses Projekt angeeignet wurden, konnten bei einigen Teammitgliedern bereits sinnvoll in ihren Unternehmen genutzt werden.


\subsection{Ausblick / Zukünftige Ziele und Funktionen}\label{subsec:ausblick-zukuenftige-ziele-und-funktionen}