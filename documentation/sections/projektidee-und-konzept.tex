\section{Projektidee und Konzept}\label{sec:projektidee-und-konzept}

\subsection{Motivation}\label{subsec:motivation}


Das Sammeln von Gegenständen ist eine Leidenschaft, die uns als Studententeam verbindet.
Von LEGO-Sets und Sammelfiguren bis hin zu Videospielen und Parfüms, egal ob aus Wertinteresse oder Selbstverwirklichungsgründen, jedes Mitglied sammelt.
Trotz dessen, dass es so viele verschiedene Sammelgegenstände und Kategorien gibt, verbindet diese Eigenschaft erstaunlich viele Menschen.
Allerdings kennen wir auch die Herausforderungen: Die Pflege mehrerer Sammlungen erfordert Zeit und Organisation.
Nach Diskussion bezüglich, was wir für die Sammlungsdokumentation benutzen, fiel dem Team schnell die Unmenge an verschiedenen speziellen Plattformen auf. \par
Obwohl es viele Plattformen zur Verwaltung einzelner Sammlungen gibt, fehlt bisher eine universelle Lösung, die es uns ermöglicht, all unsere Sammelgegenstände an einem Ort zu digitalisieren und zu organisieren.
Genau das möchten wir mit unserem Praxisprojekt ändern.
Unsere Webseite namens Collectiqo soll vorgefertigte und anpassbare Vorlagen bieten, um Sammlungen jeglicher Art zu dokumentieren und zu präsentieren.
So können wir den Pflegeaufwand reduzieren und unsere Leidenschaft noch besser genießen.


\subsection{Marktanalyse / Technologische Grundlagen}\label{subsec:marktanalyse-technologische-grundlagen}

Um Anforderungen an das Projekt zu definieren und die Plattform bestmöglich zu differenzieren, wurde zwecks diesem eine Marktanalyse zu Sammlerplattformen betrieben.
Hierbei wurden auf Aspekte wie die Zielgruppe, das Angebot auf dem Markt und einzelner Plattformen, die Benutzerfreundlichkeit und aktuelle Markttrends geachtet. \par
Die Zielgruppe bezieht sich auf Sammler verschiedener Sachgüter. \linebreak

(Umfrageergebnisse \& für Analyse Quellen und Grafiken incoming)

