\section{Durchführungsphase}\label{sec:Durchfuehrungsphase}
Die Entwicklung des Projekts lässt sich grob ich zwei Phasen aufteilen.
Die erste Phase ist die Planungsphase, in welcher die Anforderungen an das Projekt definiert und die Architektur der Anwendung festgelegt wurde.
Hierbei sind die Visualisierungen entstanden, die im vorherigen Kapitel genauer beschrieben wurden.
Die zweite Phase ist die Durchführungsphase, in welcher die Anwendung umgesetzt wurde und welche im diesem Kapitel beschrieben wird.
Hierbei wurden die Aufgaben basierend auf dem Organigramm verteilt.
Während die Verantwortlichen für die Datenbank und das Backend sich des Öfteren Abstimmen mussten, konnte das Frontend-Team weitestgehend unabhängig arbeiten.
Jedes Teammitglied hat sich selbstständig in die Thematiken eingearbeitet, die für die Umsetzung ihres Verantwortungsbereichs nötig waren.

Der übliche Ablauf während der Durchführungsphase war es, dass sich das Team einmal in der Woche zusammengesetzt hat um den aktuellen Stand der Projektbausteine zu besprechen.
Hierbei wurde abgesprochen, welche Aufgaben bis zum nächsten Treffen erledigt werden sollten.
Mussten Abschnitte aus einzelnen Bereichen zusammengeführt werden, wurde dies in einem separaten Meeting besprochen, woran nur die verantwortlichen Teammitglieder teilgenommen haben.
Die Aufgaben wurden mit YouTrack verwaltet, wodurch immer ein klares Verständnis über die Aufgabenverteilung herrschte.

\subsection{Frontend}\label{subsec:Frontend}
Lass dich aus Darko

\subsection{Backend}\label{subsec:backend}
Der erste Schritt bei der Backend entwicklung war es, einen geeigneten Startpunkt zu finden.
Hierbei wurde entschieden, dass zuerst eine Implementierung der Datenbankverbindung erfolgen sollte, da diese als Grundlage für die weitere Entwicklung dient.
Dies wurde mit der Bibliothek mysql2 realisiert, welche eine erweiterte Funktionalität gegenüber der Standardbibliothek bietet und diverse Probleme behebt.
Diese implementierung war erst möglich, nachdem die Grundstruktur der Datenbank aufgesetzt wurde.
Einen Einblick in den Code gibt Listing\ref{lst:dbconnector}.

% \lstinputlisting[language=JavaScript,label={lst:dbconnector}]{../../server/dbConnections/connectToMYSQL.js}

Als Nächstes wurde sich dazu entschieden, ein einfaches Sign Up und Login System zu programmieren.
Hierbei wurde auf die Bibliothek bcrypt zurückgegriffen, welche das Hashen von Passwörtern erleichtert.
Die SQL queries, die hier genutzt werden, wurden ebenfalls mit der mysql2 Bibliothek realisiert.
Die einzigen Nutzerdaten, die hierbei gespeichert werden, sind der Username, die E-Mail und das Passwort.
Beim Login wurde darauf geachtet, das Nutzer Username und E-Mail benutzen können, um sich einzuloggen.

Nun stellte sich die Frage, wie es möglich ist, dass Nutzer nur auf ihre eigenen Sammlungen zugreifen können.
Beim Recherchieren sind wir auf das Konzept von Sessions gestoßen und wollten diese ausprobieren.
Die erste Anwendung fanden Sessions im Speichern des Usernamens in einer Session Variable, nachdem der User sich angemeldet hat.
Der nächste Schritt war es, die Sammlungen eines Nutzers basierend auf dem Usernamen aus der Datenbank zu laden.
Zuvor muss jedoch erstmal eine Funktion implementiert werden, mit der Sammlungen angelegt werden können.
Hierzu muss zuerst die Datenbank implementiert werden, in welcher die Sammlungen gespeichert werden.
Als Datenbank für die Sammlungen wurde sich für MongoDB entschieden, welches im Kapitel~\ref{subsec:Datenbanken} genauer erläutert wird.

\subsection{Datenbanken}\label{subsec:Datenbanken}

Bei der Einrichtung der Datenbank stellt sich als erste Hürde der Umgang mit Docker Container heraus.
Zwar waren grundlegende Kentnisse über Docker vorhanden, doch eine Datenbank praktisch in einem Container hochzufahren und diese dann mit dem Code und der Programmierumgebung zu verknüpfen, war eine neue Herausforderung.
Hierzu mehr in Kapitel~\ref{sec:Herausforderungen}.
Wie in der Planung entschieden, sollen zwei Datenbanken genutzt werden - eine MySQL Datenbank für strukturierte Daten und eine MongoDB Datenbank für unstrukturierte Daten.
Im ersten Schritt wurde die MySQL Datenbank aufgesetzt und mit Tabellen für Nutzerdaten und den drei Pre-Sets für Sammlungen befüllt.
Die Nutzerdaten beinhalten Username, E-Mail und Passwort.
Für die Themenbereiche Video Spiele und Parfum wurden jeweils eine Tabelle erstellt, die für das Thema passende Spalten beinhalten.

Als Datenbank für die Sammlungen wurde sich für eine MongoDB Datenbank entschieden.
Dies liegt daran, dass MongoDB eine dokumentenorientierte Datenbank ist, die sich gut für die Speicherung von unstrukturierten Daten eignet.
Die Daten der Sammlungen sind unstrukturiert, da sie sich je nach Thema unterscheiden.
Dies ist das erste Mal, dass das Team mit einer dokumentenorientierten Datenbank arbeitet, daher musste erstmal ein Verständnis über den Aufbau einer solchen Datenbank geschaffen werden.
Zunächst wurde versucht vergleiche mit einer SQL Datenbank herzustellen, wobei schnell auffiel, dass Konzepte wie Schemata hier Collections sind.
Das Integrieren der Datenbank in der Programmierumgebung war schnell erledigt, da dies analog zu der MySQL Datenbankverbindung erfolgt ist.
Das Auslesen und schreiben von Daten in die Datenbank war dank der MongoDB Bibliothek einfach erledigt.
Funktionen für die Verbindung mit der Datenbank wurden in einer separaten Datei ausgelagert.