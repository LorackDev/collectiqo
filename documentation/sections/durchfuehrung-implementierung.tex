\section{Durchführungsphase}\label{sec:Durchfuehrungsphase}
Die Entwicklung des Projekts lässt sich grob ich zwei Phasen aufteilen.
Die erste Phase ist die Planungsphase, in welcher die Anforderungen an das Projekt definiert und die Architektur der Anwendung festgelegt wurde.
Hierbei sind die Visualisierungen entstanden, die im vorherigen Kapitel genauer beschrieben wurden.
Die zweite Phase ist die Durchführungsphase, in welcher die Anwendung umgesetzt wurde und welche im diesem Kapitel beschrieben wird.
Hierbei wurden die Aufgaben basierend auf dem Organigramm verteilt.
Während die Verantwortlichen für die Datenbank und das Backend sich des Öfteren Abstimmen mussten, konnte das Frontend-Team weitestgehend unabhängig arbeiten.
Jedes Teammitglied hat sich selbstständig in die Thematiken eingearbeitet, die für die Umsetzung ihres Verantwortungsbereichs nötig waren.

Der übliche Ablauf während der Durchführungsphase war es, dass sich das Team einmal in der Woche zusammengesetzt hat um den aktuellen Stand der Projektbausteine zu besprechen.
Hierbei wurde abgesprochen, welche Aufgaben bis zum nächsten Treffen erledigt werden sollten.
Mussten Abschnitte aus einzelnen Bereichen zusammengeführt werden, wurde dies in einem separaten Meeting besprochen, woran nur die verantwortlichen Teammitglieder teilgenommen haben.
Die Aufgaben wurden mit YouTrack verwaltet, wodurch immer ein klares Verständnis über die Aufgabenverteilung herrschte.

\subsection{Frontend}\label{subsec:Frontend}
Die Entwicklung des Webfrontends konzentrierte sich auf die Erstellung einer benutzerfreundlichen Oberfläche und alle nötigen Grundfunktionen und Strukturen.
Mithilfe von HTML, CSS und EJS wurde eine schlichte Webseite gestaltet.

Das Herzstück der Webseite ist das Anlegen eigener Collections, welche dynamisch mit Daten aus der hinterlegten MongoDB Datenbank befüllt wird.
Die Datenbank wird dabei durch JavaScript und dessen Frameworks und Bibliotheken angesprochen, welche wiederum durch das Frontend angestoßen werden.
Für das Webfrontend wurde eine Express-Anwendung in Node.js erstellt, welche auf statische Dateien (CSS, JS, Bilder) aus dem Ordnerverzeichnis zugreift.
Die Verwendung von Node.js ermöglicht das Definieren von HTTP Routen mit JavaScript, um so durch die verschiedenen Seiten der Webseite navigieren zu können und serverseitige Funktionen aufzurufen.

EJS wurde als Template-Engine verwendet, um die Integration von JavaScript innerhalb von HTMLDokumenten zu ermöglichen und somit auch seine Sammlungen abrufen zu können.
Dies erlaubte eine flexible und dynamische Gestaltung der Webseite, insbesondere durch die Verwendung von EJS-Tags wie \grqq\textless{}\%=\%\textgreater{}\grqq{}, die es ermöglichte, serverseitige Variablen oder Ausdrücke direkt in die HTML-Struktur einzufügen.
In diesem Fall wurden die Collections aus der Datenbank abgerufen und über eine Schleife in der EJS-Vorlage gerendert und werden einmal in einer Übersicht angezeigt.
In dieser Übersicht hat man die Möglichkeit eine beliebig angelegte Collection anzuklicken um so diese anzuschauen oder auch zu bearbeiten.

\subsection{Backend}\label{subsec:backend}
Der erste Schritt bei der Backend entwicklung war es, einen geeigneten Startpunkt zu finden.
Hierbei wurde entschieden, dass zuerst eine Implementierung der Datenbankverbindung erfolgen sollte, da diese als Grundlage für die weitere Entwicklung dient.
Dies wurde mit der Bibliothek mysql2 realisiert, welche eine erweiterte Funktionalität gegenüber der Standardbibliothek bietet und diverse Probleme behebt.
Diese implementierung war erst möglich, nachdem die Grundstruktur der Datenbank aufgesetzt wurde.
Einen Einblick in den Code gibt Listing\ref{lst:dbconnector}.

% \lstinputlisting[language=JavaScript,label={lst:dbconnector}]{../../server/dbConnections/connectToMYSQL.js}

Als Nächstes wurde sich dazu entschieden, ein einfaches Sign Up und Login System zu programmieren.
Hierbei wurde auf die Bibliothek bcrypt zurückgegriffen, welche das Hashen von Passwörtern erleichtert.
Die SQL queries, die hier genutzt werden, wurden ebenfalls mit der mysql2 Bibliothek realisiert.
Die einzigen Nutzerdaten, die hierbei gespeichert werden, sind der Username, die E-Mail und das Passwort.
Beim Login wurde darauf geachtet, das Nutzer Username und E-Mail benutzen können, um sich einzuloggen.

Nun stellte sich die Frage, wie es möglich ist, dass Nutzer nur auf ihre eigenen Sammlungen zugreifen können.
Beim Recherchieren sind wir auf das Konzept von Sessions gestoßen und wollten diese ausprobieren.
Die erste Anwendung fanden Sessions im Speichern des Usernamens in einer Session Variable, nachdem der User sich angemeldet hat.
Der nächste Schritt war es, die Sammlungen eines Nutzers basierend auf dem Usernamen aus der Datenbank zu laden.
Zuvor muss jedoch erstmal eine Funktion implementiert werden, mit der Sammlungen angelegt werden können.
Hierzu muss zuerst die Datenbank implementiert werden, in welcher die Sammlungen gespeichert werden.
Als Datenbank für die Sammlungen wurde sich für MongoDB entschieden, welches im Kapitel~\ref{subsec:Datenbanken} genauer erläutert wird.

Eine dieser Stellen war der Umgang mit Docker Containern.
Ein gewisses Grundverständnis war vorhanden, doch praktische Erfahrung war im gesamten Team nicht vorhanden.
Zwar was das Hochfahren einer Datenbank in einem Container schnell erreicht, doch ein Verständnis für Datenpersistenz und Volumes zu entwickeln, benötigte seine Zeit.
Auch im späteren Verlauf des Projekts, als es darum ging alle Applikationskomponenten in einer Docker-Compose Datei zusammenzuführen, stellte sich als Herausforderung heraus.
Wie sich Dockerfiles in dem ganzen System einordnen und wo sie genutzt werden, war ebenfalls neu für das Team.
Allgemein hat das Einfinden in Docker dem Team mehr Zeit als erwartet abverlangt.

Eine weitere Herausforderung war das Einfinden in die verschiedenen Bibliotheken, die bei Webentwicklung mit JavaScript genutzt werden.
Zwar wurde in der Planung bereits einige Bibliotheken festgelegt, die genutzt werden sollten, doch während der Entwicklung kamen einige neue hinzu.
Ein Beispiel hierfür ist die Bibliothek express-session, die für das Session-Management genutzt wird.
Da zur Planung noch kein detailliertes Verständnis darüber vorhanden war, welche Komponenten bei der Entwicklung einer solchen Anwendung benötigt werden, wurden Thematiken wie Session Management erst während der Entwicklung entdeckt.
Hierdurch kam es an einigen Stellen zu steilen, jedoch unvermeidbaren Lernkurven, die jedoch auch entsprechenden zeitlichen Aufwand mit sich gebracht haben.

Da das Team zuvor noch nie eine Webanwendung entwickelt hat, fehlte es an Verständnis, wie viel Komponenten Teil einer solchen Anwendung sind.
So kam es, dass während der Entwicklung immer wieder neue Komponenten entdeckt wurden, die in der Planung nicht berücksichtigt wurden.
Ein konkretes Beispiel hierfür ist das Session-Management.

\subsection{Datenbanken}\label{subsec:Datenbanken}

Bei der Einrichtung der Datenbank stellt sich als erste Hürde der Umgang mit Docker Container heraus.
Zwar waren grundlegende Kentnisse über Docker vorhanden, doch eine Datenbank praktisch in einem Container hochzufahren und diese dann mit dem Code und der Programmierumgebung zu verknüpfen, war eine neue Herausforderung.
Hierzu mehr in Kapitel~\ref{sec:Herausforderungen}.
Wie in der Planung entschieden, sollen zwei Datenbanken genutzt werden - eine MySQL Datenbank für strukturierte Daten und eine MongoDB Datenbank für unstrukturierte Daten.
Im ersten Schritt wurde die MySQL Datenbank aufgesetzt und mit Tabellen für Nutzerdaten und den drei Pre-Sets für Sammlungen befüllt.
Die Nutzerdaten beinhalten Username, E-Mail und Passwort.
Für die Themenbereiche Video Spiele und Parfum wurden jeweils eine Tabelle erstellt, die für das Thema passende Spalten beinhalten.

Als Datenbank für die Sammlungen wurde sich für eine MongoDB Datenbank entschieden.
Dies liegt daran, dass MongoDB eine dokumentenorientierte Datenbank ist, die sich gut für die Speicherung von unstrukturierten Daten eignet.
Die Daten der Sammlungen sind unstrukturiert, da sie sich je nach Thema unterscheiden.
Dies ist das erste Mal, dass das Team mit einer dokumentenorientierten Datenbank arbeitet, daher musste erstmal ein Verständnis über den Aufbau einer solchen Datenbank geschaffen werden.
Zunächst wurde versucht vergleiche mit einer SQL Datenbank herzustellen, wobei schnell auffiel, dass Konzepte wie Schemata hier Collections sind.
Das Integrieren der Datenbank in der Programmierumgebung war schnell erledigt, da dies analog zu der MySQL Datenbankverbindung erfolgt ist.
Das Auslesen und schreiben von Daten in die Datenbank war dank der MongoDB Bibliothek einfach erledigt.
Funktionen für die Verbindung mit der Datenbank wurden in einer separaten Datei ausgelagert.

Die Herausforderungen, die beim Einrichten des Datenbanksystems entstanden sind, teilen sich in zwei Punkte auf.
Der erste Punkt hängt damit zusammen, dass das Einrichten der Datenbanken gleichzeitig die erste praktische Nutzung von Docker Containern war.
Somit musste erstmal ein Verständnis dafür entwickelt werden, wie vom Hostsystem auf die Container zugegriffen werden kann und wie Daten persistiert werden.
Da die Datenbanken vorgefertigten Inhalt benötigten, wie bspw.\ die Templates für die Sammlungen in der MongoDB oder das Datenbankschema für User in der MySQL Datenbank, musste sich angeeignet werden, wie Docker Images erstellt werden können.
Darüber hinaus musste recherchiert werden, wie man solche Images hosten kann, sodass jeder beim Starten der Docker Compose Datei die Datenbanken mit den nötigen Inhalten erhält.