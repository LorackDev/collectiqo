\section{Durchführung und Implementierung}\label{sec:durchfuehrung-implementierung}

Die Entwicklung des Projekts lässt sich grob ich zwei Phasen aufteilen.
Die erste Phase, beschrieben in den vergangenen Kapiteln, ist die Planungsphase, in welcher die Anforderungen an das Projekt definiert und die Architektur der Anwendung festgelegt wurde.
Beendet wurde diese Phase mit der ersten Präsentation des Projekts, in welcher die Projektidee samt Konzept und Projektplan vorgestellt und die Erlaubnis zur Durchführung des Projekts erteilt wurde.
Die zweite Phase ist die Durchführungsphase, in welcher die Anwendung umgesetzt und welche im diesem Kapitel beschrieben wird.
Das Kapitel ist aufgeteilt in die Beschreibung der Entwicklung des Frontends in Abschnitt~\ref{subsec:entwicklung-des-webfrontends}, des Backends in~\ref{subsec:EntwicklungDesBackends}, des Datenbanksystems in~\ref{subsec:entwicklung-der-datenbank-und-datenstruktur} und die Verbindung aller Komponenten in~\ref{subsec:verbinden-des-frontends-und-backends}.
Außerdem wird auf die Implementierung von Tests in Abschnitt~\ref{subsec:implementieren-von-tests} eingegangen.

Die Aufgaben wurden basierend auf dem Organigramm verteilt.
Während die Verantwortlichen für die Datenbank und das Backend sich des Öfteren abstimmen mussten, konnte das Frontend-Team weitestgehend unabhängig arbeiten.
Jedes Teammitglied hat sich selbstständig in die Thematiken eingearbeitet, die für die Umsetzung ihres Verantwortungsbereichs nötig waren.

Der übliche Ablauf während der Durchführungsphase war es, dass wöchentliche Teammeetings stattfanden, um den aktuellen Stand der Projektbausteine zu besprechen.
Hierbei wurde abgesprochen, welche Aufgaben bis zum nächsten Treffen erledigt werden sollten.
Mussten Abschnitte aus einzelnen Bereichen zusammengeführt werden, wurde dies in einem separaten Meeting besprochen, woran nur die verantwortlichen Teammitglieder teilgenommen haben.
Die Aufgaben wurden mit YouTrack verwaltet, wodurch immer ein klares Verständnis über die Aufgabenverteilung herrschte.
Das Dashboard wurde während den Meetings basierend auf dem Stand und der Verteilung aktualisiert.
Die Aufgaben wurden in `Open', `In Progress', `To Be Reviewed' und `Done' unterteilt.
Ein Einblick in die YouTrack Aufgabenverwaltung ist in Abbildung~\ref{fig:youtrack_screenshot_agile_board} zu sehen.

\begin{figure}[h]
    \centering
    \includegraphics[width=0.9\textwidth]{youtrack_screenshot_agile_board}
    \caption{Screenshot des Agile Dashboards in YouTrack}
    \label{fig:youtrack_screenshot_agile_board}
\end{figure}

\subsection{Entwicklung des Webfrontends}\label{subsec:entwicklung-des-webfrontends}
Die Entwicklung des Webfrontends konzentrierte sich auf die Erstellung einer benutzerfreundlichen Oberfläche und alle nötigen Grundfunktionen und Strukturen.
Mithilfe von HTML, CSS und EJS wurde eine schlichte Webseite gestaltet.

Das Herzstück der Webseite ist das Anlegen eigener Collections, welche dynamisch mit Daten aus der hinterlegten MongoDB Datenbank befüllt wird.
Die Datenbank wird dabei durch JavaScript und dessen Frameworks und Bibliotheken angesprochen, welche wiederum durch das Frontend angestoßen werden.
Für das Webfrontend wurde eine Express-Anwendung in Node.js erstellt, welche auf statische Dateien (CSS, JS, Bilder) aus dem Ordnerverzeichnis zugreift.
Die Verwendung von Node.js ermöglicht das Definieren von HTTP Routen mit JavaScript, um so durch die verschiedenen Seiten der Webseite navigieren zu können und serverseitige Funktionen aufzurufen.

EJS wurde als Template-Engine verwendet, um die Integration von JavaScript innerhalb von HTML-Dokumenten zu ermöglichen und somit auch Sammlungen abrufen zu können.
Dies erlaubte eine flexible und dynamische Gestaltung der Webseite, insbesondere durch die Verwendung von EJS-Tags wie \grqq\textless{}\%=\%\textgreater{}\grqq{}, die es ermöglichten, serverseitige Variablen oder Ausdrücke direkt in die HTML-Struktur einzufügen.

Wenn der Nutzer zu Collectiqo navigiert, gelangt er zunächst auf eine Landing-Page, zu sehen in Abbildung~\ref{fig:landing-page}.

\begin{figure}[h]
    \centering
    \includegraphics[width=0.8\textwidth]{landing-page}
    \caption{Login/Sign-Up}
    \label{fig:landing-page}
\end{figure}

Diese ist recht simpel gestaltet und bietet nur wenige Interaktionsmöglichkeiten.
Der Nutzer sieht sofort, dass er sich zuerst anmelden muss, bevor er seine Sammlung anlegen kann.
Im Header befindet sich eine Schaltfläche ``Login or Register'', und in der Mitte der Landing-Page befinden sich zwei Schaltflächen ``Login'' und ``Sign-Up''.
Dadurch wird der Nutzer aufgefordert, entweder einen neuen Account zu erstellen, falls er zum ersten Mal auf dieser Seite ist, oder sich mit seinem bestehenden Account einzuloggen.
\newpage
Auf der Sign-Up-Seite befinden sich drei erforderliche Eingabefelder: Username, E-Mail und Passwort, die zur Erstellung eines Accounts benötigt werden.
Unterhalb befindet sich eine ``Sign Up''-Schaltfläche, die das Formular validiert und im Backend den Account erstellt.
Auf der Login-Seite werden Username/Email und das Passwort abgefragt, die dann mittels ``Login''-Schaltfläche im Backend validiert werden, um den Nutzer erfolgreich oder erfolglos einzuloggen.
Beide Seiten verlinken sich gegenseitig, falls doch die jeweils andere Aktion durchgeführt werden soll.
Die beiden Seiten sind in Abbildung~\ref{fig:login-sign-up-pages} dargestellt.
Nach erfolgreichem Login wird auf die ``Home''-Page weitergeleitet, siehe Abbildung~\ref{fig:collection-home-page}.

\begin{figure}[h]
    \centering
    \includegraphics[width=0.7\textwidth]{login-sign-up-pages}
    \caption{Login/Sign-Up}
    \label{fig:login-sign-up-pages}
\end{figure}

\begin{figure}[h]
    \centering
    \includegraphics[width=0.8\textwidth]{collection-home-page}
    \caption{Collection Home}
    \label{fig:collection-home-page}
\end{figure}

\newpage

Da der Nutzer bisher keine Sammlungen angelegt hat, muss er zuerst eine neue Sammlung anlegen.
Das Anlegen kann der Benutzer über einen Reiter in der Sidebar namens ``+ New Collection'' durchführen.
In diesem Fenster kann der Benutzer jetzt eine Sammlung anlegen.
Benötigt wird zuerst was für eine Sammlung man anlegen möchte.
Die Presets Video Spiele, Parfüm und Filme sind vorgefertigte Templates, die der Benutzer auswählen kann, jedoch besteht auch die möglichkeit eine eigene Kategorie anzulegen.
Ein Name für die Sammlung und eine Farbe soll der Nutzer auswählen können.
Bei einer eigenen Sammlung gibt es die Möglichkeit die einzelnen Attribute anzugeben, die man braucht, wie beispielsweise das Erscheinungsjahr eines Comics.
Zu sehen ist dieses Fenster in Abbildung~\ref{fig:collection-add-modal}.

Hat der Nutzer jetzt eine Sammlung angelegt, wird er wieder auf seine Home-Seite geleitet.
Auf dieser Seite erscheint eine Liste.
In dieser Liste wird jetzt die neue Sammlung angezeigt, zukünftig werden alle Sammlungen zentral hier erreichbar sein.
Hier gibt es die Möglichkeit entweder mit einem Klick auf den Sammlungsnamen die Sammlung aufzurufen oder seine Sammlung zu löschen.
Angenommen der Nutzer möchte seine Sammlung pflegen, so klickt er auf gewünschte angelegte Sammlung.
In einem neuen Fenster hat er die übersicht seiner Sammlung in einer Tabelle.
Der Nutzer kann hier den Namen der Sammlung und die jeweiligen Einträge einsehen und bearbeiten (siehe Abbildung~\ref{fig:collection-home-page2}).
\newpage
\begin{figure}[h]
    \centering
    \includegraphics[width=0.95\textwidth]{collection-add-modal}
    \caption{Sammlung anlegen}
    \label{fig:collection-add-modal}
\end{figure}

Der Aufbau der Home-Seite des Nutzer befindet sich in der home.ejs.
Einen Ausschnitt aus der home.ejs mit den HTML und EJS Tags findet sich im folgenden Code:

\vspace{1em}
\lstinputlisting[language=JavaScript,firstline=26, lastline=54, label={lst:homeejs}]{../views/pages/home.ejs}
\vspace{1em}

Einen Ausschnitt des CSS für home.ejs findet sich in folgendem Code:

\vspace{1em}
\lstinputlisting[language=css,firstline=11, lastline=50, label={lst:homecss}]{../public/css/home.css}
\vspace{1em}

Nach dem anlegen einer neuen Sammlung wird der Nutzer wieder auf seine Home-Seite weitergeleitet.
In dieser ist die Sammlung nach dem anlegen aufgelistet.
Möchte der Nutzer nun seine Sammlung einsehen kann er auf den Namen der Sammlung klicken, was dazu führt, dass eine neue Seite geladen wird auf der die angelegte Sammlung angezeigt wird.

\begin{figure}[h]
    \centering
    \includegraphics[width=0.95\textwidth]{collection-show-data}
    \caption{Collection Ansicht}
    \label{fig:collection-show-data}
\end{figure}

Die weiterleitung auf diese Seite wird durch ein JavaScript aus der home.ejs angestoßen.
Das Script befindet sich in folgendem Code:

\vspace{1em}
\lstinputlisting[language=JavaScript,firstline=58, lastline=68, label={lst:appjs}]{../views/pages/home.ejs}
\vspace{1em}

In der Sidebar der seite befinden sich unterschieldiche Reiter, davon ist einer mit Settings betitel.
Klickt der Nutzer auf diesen reiter wird er auf eine neue Seite weitergeleitet, bei denen man Einstellungen zu seinem Account tätigen kann.
Darunter zählt beispielsweise das ändern des Account-Passworts, ändern des Nutzernamens oder der E-Mail.
Die Seite ist strukturell fertig, die Funktionen dafür sind allerdings noch nicht implementiert.

\begin{figure}[h]
    \centering
    \includegraphics[width=0.95\textwidth]{usersettings}
    \caption{Account Einstellungen}
    \label{fig:usersettings}
\end{figure}

\newpage

\subsection{Entwicklung des Backends}\label{subsec:EntwicklungDesBackends}
Der erste Schritt bei der Backend entwicklung war es, einen geeigneten Startpunkt zu finden.
Hierbei wurde entschieden, dass zuerst eine Implementierung der Datenbankverbindung erfolgen sollte, da diese als Grundlage für die weitere Entwicklung dient.
Dies wurde mit der Bibliothek mysql2 realisiert, welche eine erweiterte Funktionalität gegenüber der Standardbibliothek bietet und diverse Probleme behebt.
Diese implementierung war erst möglich, nachdem die Grundstruktur der Datenbank aufgesetzt wurde.
Einen Einblick gibt der folgende Code:

\vspace{1em}
\lstinputlisting[language=JavaScript,label={lst:mysqlconnection}]{../server/dbConnections/connectToMYSQL.js}
\vspace{1em}
\newpage

Als Nächstes wurde sich dazu entschieden, ein einfaches Sign Up und Login System zu programmieren.
Hierbei wurde auf die Bibliothek bcryptjs zurückgegriffen, welche das Hashen von Passwörtern erleichtert.
Die SQL queries, die hier genutzt werden, wurden ebenfalls mit der mysql2 Bibliothek realisiert.
Die einzigen Nutzerdaten, die hierbei gespeichert werden, sind der Username, die E-Mail und das Passwort.
Beim Login wurde darauf geachtet, das Nutzer Username und E-Mail benutzen können, um sich einzuloggen.
Einen Einblick gibt der folgende Code:

\vspace{1em}
\lstinputlisting[languange=JavaScript,label={lst:login}]{../server/authentication/loginHandler.js}
\vspace{1em}

Nun stellte sich die Frage, wie es möglich ist, dass Nutzer nur auf ihre eigenen Sammlungen zugreifen können.
Beim Recherchieren sind wir auf das Konzept von Sessions gestoßen und wollten diese ausprobieren.
Die erste Anwendung fanden Sessions im Speichern des Usernamens in einer Session Variable, nachdem der User sich angemeldet hat.
Der nächste Schritt war es, die Sammlungen eines Nutzers basierend auf dem Usernamen aus der Datenbank zu laden.
Zuvor muss jedoch erstmal eine Funktion implementiert werden, mit der Sammlungen angelegt werden können.
Hierzu muss zuerst die Datenbank implementiert werden, in welcher die Sammlungen gespeichert werden.
Als Datenbank für die Sammlungen wurde sich für MongoDB entschieden, welches im Kapitel~\ref{subsec:entwicklung-der-datenbank-und-datenstruktur} genauer erläutert wird.

Nur das Anlegen von Sammlungen reicht natürlich nicht, weshalb weitere Funktionen im Backend implementiert wurden, die diverse grundlegende Funktionen ermöglichen.
Hierzu zählen das Löschen von Sammlungen, das Hinzufügen von Items zu Sammlungen und das Löschen von Items aus Sammlungen.
Außerdem wurde eine separate Funktion implementiert, die das Erstellen einer Sammlung direkt von einem Template aus ermöglicht.
Das Hinzufügen von Daten in eine existierende Sammlung wird mit dem folgenden Code ermöglicht:

\vspace{1em}
\lstinputlisting[language=JavaScript, label={lst:addCollectionEntry}]{../server/collections/addCollectionEntry.js}
\vspace{1em}

Während der Entwicklung des Backends kam es jedoch auch des Öfteren zu Herausforderungen.
Eine dieser Stellen war der Umgang mit Docker Containern.
Ein gewisses Grundverständnis war vorhanden, doch an praktischer Erfahrung fehlte es dem Team.
Zwar was das Hochfahren einer Datenbank in einem Container schnell erreicht, doch ein Verständnis für Datenpersistenz und Volumes zu entwickeln, benötigte seine Zeit.
Auch im späteren Verlauf des Projekts, als es darum ging alle Applikationskomponenten in einer Docker-Compose Datei zusammenzuführen, stellte sich als Herausforderung heraus.
Wie sich Dockerfiles in dem ganzen System einordnen und wo sie genutzt werden, war ebenfalls neu für das Team.
Allgemein hat das Einfinden in Docker dem Team mehr Zeit als erwartet abverlangt.
Das Image für die MySQL Datenbank ist selbst erstellt, basierend auf dem offiziellen MySQL Image.
Diese beinhalten beispieleinträge für diverse Sammlungen verschiedener Nutzer, bereits angelegte Testnutzer und die Datenbankstruktur inklusive der Templates für die Sammlungen.
Am Ende sah die Docker-Compose Datei wie im folgenden Code aus:

\vspace{1em}
\lstinputlisting[language=docker-compose, label={lst:docker-compose}]{../docker-compose.yaml}
\vspace{1em}

Das Docker Image für die NodeJS Applikation wurde ebenfalls selbst erstellt, basierend auf dem offiziellen NodeJS Image.
Die dazugehörige Dockerfile sieht wie folgt aus:

\vspace{1em}
\lstinputlisting[language=dockerfile, label={lst:dockerfile}]{../Dockerfile}
\vspace{1em}

Eine weitere Herausforderung war das Einfinden in die verschiedenen Bibliotheken, die bei Webentwicklung mit JavaScript genutzt werden.
Zwar wurde in der Planung bereits einige Bibliotheken festgelegt, die genutzt werden sollten, doch während der Entwicklung kamen einige neue hinzu.
Ein Beispiel hierfür ist die Bibliothek express-session, die für das Session-Management genutzt wird.
Da zur Planung noch kein detailliertes Verständnis darüber vorhanden war, welche Komponenten bei der Entwicklung einer solchen Anwendung benötigt werden, wurden Thematiken wie Session Management erst während der Entwicklung entdeckt.
Hierdurch kam es an einigen Stellen zu steilen, jedoch unvermeidbaren Lernkurven, die jedoch auch entsprechenden zeitlichen Aufwand mit sich gebracht haben.

Da das Team zuvor noch nie eine Webanwendung entwickelt hat, fehlte es an Verständnis, wie viele Komponenten Teil einer solchen Anwendung sind.
So kam es, dass während der Entwicklung immer wieder neue Komponenten entdeckt wurden, die in der Planung nicht berücksichtigt wurden.
Ein konkretes Beispiel hierfür ist das Session-Management.

\subsection{Entwicklung der Datenbank und Datenstruktur}\label{subsec:entwicklung-der-datenbank-und-datenstruktur}

Bei der Einrichtung der Datenbank stellt sich als erste Hürde der Umgang mit Docker Container heraus.
Zwar waren grundlegende Kenntnisse über Docker vorhanden, doch eine Datenbank praktisch in einem Container hochzufahren und diese dann mit dem Code und der Programmierumgebung zu verknüpfen, war eine neue Herausforderung.
Wie in der Planung entschieden, sollen zwei Datenbanken genutzt werden - eine MySQL Datenbank für strukturierte Daten und eine MongoDB Datenbank für unstrukturierte Daten.
Im ersten Schritt wurde die MySQL Datenbank aufgesetzt und mit einer Tabelle für Nutzerdaten befüllt.
Die Nutzerdaten beinhalten Username, E-Mail und Passwort.
Da diese Tabelle eine feste Form besitzt, wurde sich für eine relationale Datenbank entschieden.
Die Struktur der MySQL Datenbank ist in Abbildung~\ref{fig:mysql_structure} zu sehen.

\newpage
\begin{figure}[h]
    \centering
    \includegraphics[width=0.4\textwidth]{mysql_structure}
    \caption{Struktur der MySQL Datenbank}
    \label{fig:mysql_structure}
\end{figure}

Die Datenbank wurde mit einigen Testnutzern befüllt, mithilfe zweier SQL Queries.
Die erste Query erstellt die Tabelle für die Nutzerdaten:

\vspace{1em}
\lstinputlisting[language=SQL, label={lst:createDatabase}]{../scripts/mysql/create_database.sql}
\vspace{1em}

Die zweite Query fügt die Testnutzer hinzu:

\vspace{1em}
\lstinputlisting[language=SQL, label={lst:insertValues}]{../scripts/mysql/insert_values.sql}
\vspace{1em}

Als Datenbank für die Sammlungen und die Vorlagen wurde sich für eine MongoDB Datenbank entschieden.
Dies liegt daran, dass MongoDB eine dokumentenorientierte Datenbank ist, die sich gut für die Speicherung von unstrukturierten Daten eignet.
Die Daten der Sammlungen sind unstrukturiert, da sie sich je nach Thema unterscheiden, da jede Sammlung andere Spalten besitzt.
Dasselbe gilt für die Vorlagen, die zum Erstellen von Sammlungen genutzt werden können.
Dies ist das erste Mal, dass das Team mit einer dokumentenorientierten Datenbank arbeitet, daher musste erstmal ein Verständnis über den Aufbau einer solchen Datenbank geschaffen werden.
Zunächst wurde versucht Vergleiche mit einer SQL Datenbank herzustellen, wobei schnell auffiel, dass Konzepte wie Schemata hier Collections sind.
Für dieses Projekt wurden zwei verschiedene Collections angelegt, eine für die Vorlagen, die zum Erstellen eigener Sammlungen genutzt werden können und eine für die Sammlungen selbst.
Die Struktur der MongoDB Datenbank ist in Abbildung~\ref{fig:mongodb_structure} zu sehen.

\begin{figure}[h]
    \centering
    \includegraphics[width=0.4\textwidth]{mongodb_structure}
    \caption{Struktur der MongoDB Datenbank}
    \label{fig:mongodb_structure}
\end{figure}

Die Daten innerhalb der Collection für die Sammlungen sieht wie in Abbildung~\ref{fig:mongodb_collections} aus.
Die Sammlung für Videospiele wurde hierbei anhand einer Vorlage aus dem Vorlagenkatalog erstellt.

\begin{figure}[h!]
    \centering
    \includegraphics[width=1\textwidth]{mongodb_collections}
    \caption{Daten innerhalb der MongoDB Collection}
    \label{fig:mongodb_collections}
\end{figure}

Einzelne Einträge innerhalb der Collections werden wie in Abbildung~\ref{fig:mongodb_collections_entries} strukturiert.
Die Spalten sind exemplarisch für eine Autosammlung und ändern sich je nach Thematik.

\begin{figure}[h!]
    \centering
    \includegraphics[width=0.35\textwidth]{mongodb_collections_entries}
    \caption{Einträge innerhalb einer Sammlung}\label{fig:mongodb_collections_entries}
\end{figure}
\newpage

Die Vorlagen für die Sammlungen sind in Abbildung~\ref{fig:mongodb_templates} zu sehen.
Aktuell sind diese festgelegt, ohne Option diese zu verändern oder neue hinzuzufügen und sie sind gültig für alle User der Website.
In späteren Versionen soll jedoch das Anlegen von eigenen Vorlagen möglich sein, welche dann auch einzelnen Nutzern zugewiesen werden.
Bringt man einen Community-Aspekt mit ein, so könnte es in zukunft möglich sein, diese Vorlagen untereinander zu teilen.

\begin{figure}[h]
    \centering
    \includegraphics[width=1\textwidth]{mongodb_templates}
    \caption{Daten innerhalb der MongoDB Collection für die Vorlagen}
    \label{fig:mongodb_templates}
\end{figure}
\newpage

Befüllt wurde die MongoDB Datenbank mit einem Skript, welches die Vorlagen und Beispieldaten für die Sammlungen einfügt.
Das Skript für die Vorlagen sieht so aus:

\vspace{1em}
\lstinputlisting[language=JavaScript, label={lst:insertTemplates}]{../scripts/mongodb/init.js}
\vspace{1em}

Das Integrieren der Datenbank in der Programmierumgebung war schnell erledigt, da dies analog zu der MySQL Datenbankverbindung erfolgt ist.
Das Auslesen und schreiben von Daten in die Datenbank war dank der MongoDB Bibliothek einfach erledigt.
Funktionen für die Verbindung mit der Datenbank wurden in einer separaten Datei ausgelagert.

Die Herausforderungen, die beim Einrichten des Datenbanksystems entstanden sind, teilen sich in zwei Punkte auf.
Der erste Punkt hängt damit zusammen, dass das Einrichten der Datenbanken gleichzeitig die erste praktische Nutzung von Docker Containern war.
Somit musste erstmal ein Verständnis dafür entwickelt werden, wie vom Hostsystem auf die Container zugegriffen werden kann und wie Daten persistiert werden.
Da die Datenbanken vorgefertigten Inhalt benötigten, wie bspw.\ die Templates für die Sammlungen in der MongoDB oder das Datenbankschema für User in der MySQL Datenbank, musste sich angeeignet werden, wie Docker Images erstellt werden können.
Darüber hinaus musste recherchiert werden, wie man solche Images hosten kann, sodass jeder beim Starten der Docker Compose Datei die Datenbanken mit den nötigen Inhalten erhält.
Die Herausforderungen ähneln somit denen, die bereits bei der Entwicklung des Backends aufgetreten sind.

\subsection{Verbinden des Frontends und Backends}\label{subsec:verbinden-des-frontends-und-backends}

Der erste Schritt beim Verbinden von Front- und Backend war es, das Login und Sign Up System zu verbinden.
Nach anfänglichen Problemen mit Umgebungsvariablen, die nicht korrekt geladen wurden aufgrund einer Umstrukturierung der Projektdokumentation, war der Prozess relativ einfach.
Es musste recherchiert werden, wie die Backendfunktionen seitens des Frontends aufgerufen werden, welches durch HTTP Requests realisiert wurde, die die Routen in der app.js ansprechen.
Da sich die Entwickler des Front- und Backends während der getrennten Entwicklung regelmäßig abgesprochen haben, konnten die Funktionen schnell miteinander verbunden werden.

Problematischer wurde die Verbindung bei Funktionen, die mit der MongoDB interagieren und für das Erstellen und Bearbeiten von Sammlungen verantwortlich sind.
Eine Hürde war hierbei herauszufinden, wie Code in einer NodeJS Applikation gedebugt wird, dass Breakpoints auch genutzt werden, wenn die Funktion nur über eine Route in der app.js aufgerufen wird.
Es war dem Team nicht sofort klar, das die gesamte app.js im Debug Modus gestartet werden muss, und nicht nur die zu testende Funktion.
Mit Debugging konnten nun die Fehler herausgestellt werden, die die Verbindung verhinderten.
Neben einer falschen Extraktion der Daten aus dem Frontend lag das Problem an einer fehlerhaften Implementierung der Username Session Variable.
Nach der Lösung dieses Problems konnten nun die Sammlungen eines Users angezeigt werden.

Als Nächstes wurde die Funktion implementiert, dass ein User eine Sammlung über die Benutzeroberfläche erstellen kann.
Aktuell wird nur das Anlegen von Sammlungen über Vorlagen unterstützt, backendfunktionen für das Anlegen von komplett selbsterstellten Sammlungen existieren jedoch bereits.
Da nun Sammlungen angelegt werden können und diese auch im Frontend angezeigt werden, wurde als Nächstes die Funktion implementiert, dass diese Sammlungen auch gelöscht werden können.
Als letzte Verbindung zwischen Front- und Backend wurde implementiert, das User die Inhalte ihrer Sammlungen durch das Frontend einsehen können.
Die jeweiligen Screenshots finden sich im Kapitel~\ref{subsec:entwicklung-des-webfrontends}

Die Schnittstelle von Front- und Backend findet sich in der app.js, in welcher die Routen definiert sind.
Einen Ausschnitt aus der app.js findet sich im folgenden Code:

\vspace{1em}
\lstinputlisting[language=JavaScript,firstline=39, lastline=83, label={lst:appjs}]{../app.js}
\vspace{1em}
\newpage

\subsection{Implementieren von Tests}\label{subsec:implementieren-von-tests}

Um Teile des Backends zu testen, bevor diese mit dem Frontend verbunden werden konnten, wurde versucht Unit-Tests zu schreiben.
Dieser Prozess stellte sich als komplizierter heraus als gedacht, da das Team zuvor noch nie mit Tests für JavaScript gearbeitet hat.
So konnten zwar einfache Unit-Tests für das Login System geschrieben werden, jedoch Integrationstests, in welchen die Datenbank angesprochen wird, stellten sich als schwierig heraus.
Ein exemplarischer Test für das Login System ist in Listing~\ref{lst:loginTest} zu sehen.

\vspace{1em}
\lstinputlisting[language=JavaScript, firstline=15, lastline=32, label={lst:loginTest}]{../tests/loginHandler.test.js}
\vspace{1em}

Auch gab es Probleme damit, eine Datei für Umgebungsvariablen für Testdatenbanken korrekt in die Tests mit einzubinden.
Erst nach längerer Recherche stellte sich heraus, dass Umgebungsvariablen für jest Tests in einer separaten Config Datei definiert werden müssen und nicht wie die normale .env Datei in den Code eingebunden werden konnte.
Auch mit den Mock Funktionen von jest musste sich erst eingearbeitet werden.
An dieser Stelle wurde sich dazu entschieden, die Implementierung von Tests für alle Backend Funktionen abzubrechen, da der Aufwand zu groß war und die Zeit zu knapp.
Der zeitliche Aufwand für das Einrichten der automatisierten Tests war zu hoch, vor allem im Vergleich zu dem zeitlichen Aufwand den ein manueller Test benötigt hätte.
Manuelle Tests wurden durchgeführt, als das Backend mit dem Frontend verbunden war.
Ein Nachteil, der dadurch entstanden ist, ist dass Funktionen öfters nachträglich angepasst werden mussten, was die Verbindung zum Frontend zeitaufwendiger machte.
