\documentclass[a4paper, 12pt]{article}
\usepackage[utf8]{inputenc}
\usepackage[T1]{fontenc}
\usepackage[ngerman]{babel}
\usepackage{geometry}
\usepackage{listings}
\usepackage{xcolor}
\usepackage{hyperref}
\usepackage{helvet}
\usepackage{titlesec}

\geometry{a4paper, left=1.5cm, right=1.5cm, top=2cm, bottom=2cm}

% Einstellungen für Code-Listings
\lstset{
    basicstyle=\ttfamily,
    breaklines=true,
    numbers=left,
    numberstyle=\tiny,
    frame=single,
    backgroundcolor=\color{lightgray},
    captionpos=b,
    language=Java % Ändere dies entsprechend deiner Programmiersprache
}

\title{
    Projektdokumentation\\
    Hochschule Mainz - Wi Dual - Praxismodul I
}

\author{
    Bindernagel, Lorenz\\
    Schäfer, Robin\\
    Simic, Darko\\
    Struve, Anika
}

\date{
    \today
}



\begin{document}

    \maketitle
    \tableofcontents
    \newpage

    \section{Projektidee/ -konzeption}
    
    \subsection{Motivation}
    Design Science: Relevance-Cycle

    \subsection{Marktanalyse / Technologische Grundlagen}
    Zielgruppe, Marktabgrenzung, theoretische Grundlagen, Literaturüberblick\linebreak
    Design Science: Relevance-Cycle

    \newpage

    \section{Projektplanung / Projektmanagement}
    
    \subsection{Zieldefinition}
    
    \subsection{PSP}
    
    \subsection{Organigramm / Team}
    
    \subsection{Ablaufplanung (Gantt / Netzplan)}
    Wird in YouTrack erstellt.
    
    \subsection{Stakeholderanalyse / Risikoanalyse}
    
    \subsection{Kosten- und Aufwandsplanung}

    \newpage
    
    \section{Projektbericht}
    
    \subsection{Allgemeine Methodik / Vorgehen / Literaturüberblick}
    Bspw.: Business Model Canvas / CRSIP-DM-Cycle / Behavior Driven Development / SCRUM)
    Design Science: Rigor-Cycle
    
    \subsection{Designphase}
    Datenmodell / Frontend / Backend
    Design Science: Design-Cycle
    
    \subsection{Prototypvorstellung}
    Design Science: Artefakt

    \newpage
    
    \section{Ergebnis und Fazit}
    
    \subsection{Fazit}
    
    \subsection{Lessons learned}
    
    \subsection{Ausblick}

    \appendix

    \section{Anhang}
    
    \subsection{Testeintrag}
   

\end{document}
