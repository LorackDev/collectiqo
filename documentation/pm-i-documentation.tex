\documentclass[a4paper, 12pt]{article}
\usepackage[utf8]{inputenc}
\usepackage[T1]{fontenc}
\usepackage[ngerman]{babel}
\usepackage{geometry}
\usepackage{listings}
\usepackage{xcolor}
\usepackage{hyperref}

% Einstellungen für Seitenränder, Schriftgröße, etc.
\geometry{a4paper, left=2.5cm, right=2.5cm, top=3cm, bottom=3cm}
\setlength{\parindent}{0pt}
\setlength{\parskip}{6pt}

% Einstellungen für Code-Listings
\lstset{
    basicstyle=\ttfamily,
    breaklines=true,
    numbers=left,
    numberstyle=\tiny,
    frame=single,
    backgroundcolor=\color{lightgray},
    captionpos=b,
    language=Java % Ändere dies entsprechend deiner Programmiersprache
}

\title{Softwaredokumentation für [Dein Projekt]}
\author{[Dein Name]}
\date{\today}

\begin{document}

    \maketitle
    \tableofcontents
    \newpage

    \section{Changed for test}
    Hier kannst du eine kurze Einleitung zu deinem Projekt und der Softwaredokumentation geben.
    
    \subsection{Darko übernimmt alla}

    \section{Systemanforderungen}
    Definiere die Hardware- und Softwareanforderungen für die Ausführung deiner Software.

    \subsection{Robin erste Subsection}

    \section{Installation}
    Beschreibe den Installationsprozess deiner Software.

    \section{Benutzerhandbuch}
    Erkläre, wie Benutzer deine Software verwenden können. Dies kann Schritt-für-Schritt-Anleitungen, Screenshots usw. beinhalten.

    \section{Architektur}
    Gib einen Überblick über die Architektur deiner Software. Verwende Diagramme, um die verschiedenen Komponenten zu zeigen.

    \section{Datenmodell}
    Beschreibe die Datenstrukturen und ihre Beziehungen in deiner Software.

    \section{Programmier- und Designentscheidungen}
    Erkläre wichtige Entscheidungen, die du während des Entwicklungsprozesses getroffen hast, und warum.

    \section{API-Dokumentation (optional)}
    Wenn deine Software eine API hat, dokumentiere sie hier.

    \section{Fehlerbehebung}
    Liste häufige Probleme auf und gib Lösungen an.

    \section{Änderungsprotokoll}
    Halte alle Änderungen an deiner Software in einem Protokoll fest.

    \section{Quellcode-Dokumentation}
    Dokumentiere wichtige Teile des Quellcodes. Verwende Code-Listings, um Beispiele zu zeigen.

    \section{Schlusswort}
    Fasse die wichtigsten Punkte zusammen und gehe auf zukünftige Entwicklungspläne ein.

\end{document}
