\documentclass[a4paper, 12pt]{article}
\usepackage[utf8]{inputenc}
\usepackage[T1]{fontenc}
\usepackage[ngerman]{babel}
\usepackage{geometry}
\usepackage{listings}
\usepackage{xcolor}
\usepackage{hyperref}
\usepackage{helvet}
\usepackage{titlesec}
\usepackage{setspace}
\usepackage{graphicx}
\usepackage{pdfpages}
\usepackage{fancyhdr}
\usepackage{float}
\usepackage{enumitem}
\usepackage{tikz}


% Einstellungen für Seitenränder, Schriftgröße, etc.
\geometry{a4paper, left=1.5cm, right=1.5cm, top=2cm, bottom=2cm}
\graphicspath{ {./images/} }

% Folgende 5 Zeilen sind im BPM-Dokument zu finden
 %\pagestyle{fancy} % Add this line
 %\fancyhf{} % Add this line
 %\fancyfoot[C]{\thepage} % Add this line
 %setlength{\parindent}{0pt}
 %\setlength{\parskip}{6pt}


% Einstellungen für Code-Listings
\lstset{
    basicstyle=\ttfamily,
    breaklines=true,
    numbers=left,
    numberstyle=\tiny,
    frame=single,
    backgroundcolor=\color{lightgray},
    captionpos=b,
    language=Java % Ändere dies entsprechend deiner Programmiersprache
}

\title{
        {\LARGE Projektdokumentation - Praxismodul I}\\
        {\LARGE collectiqo}\\
        {\small Hochschule Mainz}\\
        {\vspace{-0.5em}\small University of Applied Sciences}\\
        {\vspace{-0.5em}\small Fachbereich Wirtschaft}\\
        {\small Wirtschaftsinformatik B.Sc. dual}
}

\author{
    Bindernagel, Lorenz\\
    Schäfer, Robin\\
    Simic, Darko\\
    Struve, Anika
}

% \date{\today}
 \date{16.06.2024}

% ToDo's TitlePage
% HS-Mainz Logo
% Keine Nummerierung TitlePage


% Document
\begin{document}

    %Robin Versuch ein Logo in die obere-rechte Ecke auf der Titelseite zu bekommen - aktuell noch nicht erfolgreich
\begin{titlepage} %Robin Verwendung Package "tikz"
    \begin{tikzpicture}[remember picture, overlay]
        \node[anchor=north east, inner sep=0pt] at (current page.north east) {\includegraphics[width=8cm]{hsm_logo_orange}};
    \end{tikzpicture}

    \maketitle
\end{titlepage}

    % \maketitle
    \newpage
    \tableofcontents
    \newpage


    \section{Projektidee und Konzept}\label{sec:idee-konzept}
    \subsection{Motivation}\label{subsec:Motivation}
Design Science: Relevance-Cycle
Die Motivation des Projekt ist es, einen Plattform für Sammler zu schaffen, die in verschiedenen Themenbereichen Sammlungen besitzen.


\subsection{Marktanalyse / Technologische Grundlagen}\label{subsec:Marktanalyse-TechnologischeGrundlagen}
Zielgruppe, Marktabgrenzung, theoretische Grundlagen, Literaturüberblick\linebreak
Design Science: Relevance-Cycle
    \newpage


    \section{Projektplanung und Projektmanagement}\label{sec:planung-management}
    \subsection{Zieldefinition}\label{subsec:Zieldefinition}

% Wir könnten hier sicherlich auch das Wording Meilensteine nutzen, finde ich besser als Teilergebnisse
\subsection{Teilergebnisse}\label{subsec:Teilergebnisses}
Die Teilergebnisse lassen sich aus dem Konzept in ~\ref{subsec:Designphase} herleiten.
Grundlegende Teilergebnisse sind die Erstellung des Frontends, die Erstellung des Backends und das Aufsetzen einer Datenbank.
Neben diesen Zielen, die sich aus der genutzten Architektur ergeben, ist ein weiteres Teilergebnis das Hosten der Website und der Datenbank.
Für jedes Teilergebnis sind Voranalysen nötig, da die Programmierung einer Webanwendung für das gesamte Team neu ist.
Als Startpunkt für die Analyse diente GitHub Eduction, welches mit dem Kurs Intro to Web Development einige Tools zum Entwickeln von Websites zur verfügung stellt.
Die Tools, für die sich durch die Analyse entschieden wurde, werden in ~\ref{subsec:Tools} aufgeführt.

Die Teilergebnisse sind eng miteinander verbunden.
Essentiell sind hierbei das Hosten der Website und der Datenbank.
Besonders ohne die Datenbank lässt sich das Backend schwierig entwickeln.
Das Hosten der Website ist wichtig, um eingebaute Funktionen direkt live testen zu können.
Die Entwicklung des Frontends, des Backends und der Datenbank ist eng miteinander verzahnt.
Während Frontend und die Datenbank getrennt voneinander eingerichtet werden können, ist zur Kommunikation der beiden Bereiche das Backend essentiell.
Während die drei Bereiche einzelne Teilergebnisse bilden, findet der Großteil der Programmierung dieser parallel statt.

Der Umgang des Frontends ist alles, womit der Nutzer am Ende auf der Website interagieren kann.
Beispiele hierfür sind die Login-Page, die Ansicht der verschiedenen Sammlungen und die Seite zum Erstellen von Vorlagen.
Im Anhang befinden sich Mockups, die beim initialen Brainstorming der Projektidee entstanden sind.
Das Backend soll für die Kommunikation zwischen Frontend und Datenbank verantwortlich sein.
Es soll die Daten aus der Datenbank an die Benutzeroberfläche übermitteln um diese anzuzeigen.
Darüber hinaus soll es Änderungen an den Daten, die in der Benutzeroberfläche durchgeführt werden, an die Datenbank kommunizieren.
Hierbei ist wichtig, dass die Funktionen sicher stellen, dass die Datenbank dynamisch basierend auf Nutzereingaben skaliert wird.
Die Logik der Website soll in diesem Bereich des Programms festgehalten werden.
Das Datenbanksystem ist der Ort, an dem die Informationen über Nutzer und Sammlungen hinterlegt werden.
Die Datenbankform der Vorlagen der Sammlungen werden hier gespeichert.
Sie muss dabei so eingerichtet sein, dass sie dynamisches hinzufügen und löschen ganzer Tabellen erlaubt.
% Vllt mal ein Mockup für ein Template erstellen

\subsection{PSP}\label{subsec:PSP}

\subsection{Organigramm / Team}\label{subsec:Organigramm}

\begin{figure}[htbp]
    \centering
    \includegraphics[width=0.8\textwidth]{organigramm}
    \caption{Organigramm}
\end{figure}

\subsection{Ablaufplanung (Gantt / Netzplan)}\label{subsec:Ablaufplan}
Wird in YouTrack erstellt.

\subsection{Stakeholderanalyse / Risikoanalyse}\label{subsec:Stakeholder-Risikoanalyse}

\subsection{Kosten- und Aufwandsplanung}\label{subsec:Kosten-Aufwandsplanung}

\subsection{Tools}\label{subsec:Tools}

    \newpage


    \section{Projektbericht}\label{sec:projektberichts}
    \subsection{Allgemeine Methodik / Vorgehen / Literaturüberblick}\label{subsec:Methodik}
Bspw.: Business Model Canvas / CRSIP-DM-Cycle / Behavior Driven Development / SCRUM)
Design Science: Rigor-Cycle

\subsection{Designphase}\label{subsec:Designphase}
Datenmodell / Frontend / Backend
Design Science: Design-Cycle

\subsubsection{Backend}
Der erste Schritt bei der Backend entwicklung war es, einen geeigneten Startpunkt zu finden.
Hierbei wurde entschieden, dass zuerst eine Implementierung der Datenbankverbindung erfolgen sollte, da diese als Grundlage für die weitere Entwicklung dient.
Dies wurde mit der Bibliothek mysql2 realisiert, welche eine erweiterte Funktionalität gegenüber der Standardbibliothek bietet und diverse Probleme behebt.
Diese implementierung war erst möglich, nachdem die Grundstruktur der Datenbank aufgesetzt wurde.
Einen Einblick in den Code gibt Listing\ref{lst:dbconnector}.

% \lstinputlisting[language=JavaScript,label={lst:dbconnector}]{../../server/dbConnections/connectToMYSQL.js}

Als Nächstes wurde sich dazu entschieden, ein einfaches Sign Up und Login System zu programmieren.
Hierbei wurde auf die Bibliothek bcrypt zurückgegriffen, welche das Hashen von Passwörtern erleichtert.
Die SQL queries, die hier genutzt werden, wurden ebenfalls mit der mysql2 Bibliothek realisiert.
Die einzigen Nutzerdaten, die hierbei gespeichert werden, sind der Username, die E-Mail und das Passwort.
Beim Login wurde darauf geachtet, das Nutzer Username und E-Mail benutzen können, um sich einzuloggen.

Nun stellte sich die Frage, wie es möglich ist, dass Nutzer nur auf ihre eigenen Sammlungen zugreifen können.
Beim Recherchieren sind wir auf das Konzept von Sessions gestoßen und wollten diese ausprobieren.

Parallel wurde eine Funktion entwickelt, die Daten aus der Datenbank anhand des Tabellennamens ausliest.
Diese werden dann in eine Form gebracht, welche das Frontend nutzen kann, um Beispieldaten in einer Tabelle einzupflegen.



\subsubsection{Datenbank}

Bei der Einrichtung der Datenbank stellt sich als erste Hürde der Umgang mit Docker Container heraus.
Zwar waren grundlegende Kentnisse über Docker vorhanden, doch eine Datenbank praktisch in einem Container hochzufahren und diese dann mit dem Code und der Programmierumgebung zu verknüpfen, war eine neue Herausforderung.
Hierzu mehr in Kapitel~\ref{subsec:Herausforderungen}.
Wie in der Planung entschieden, sollen zwei Datenbanken genutzt werden - eine MySQL Datenbank für strukturierte Daten und eine MongoDB Datenbank für unstrukturierte Daten.
Im ersten Schritt wurde die MySQL Datenbank aufgesetzt und mit Tabellen für Nutzerdaten und den drei Pre-Sets für Sammlungen befüllt.
Die Nutzerdaten beinhalten Username, E-Mail und Passwort.
Für die Themenbereiche Video Spiele und Parfum wurden jeweils eine Tabelle erstellt, die für das Thema passende Spalten beinhalten.

\subsection{Herausforderungen}\label{subsec:Herausforderungen}

Bei der Programmierung des Projekts gab es einige Stellen, an denen das Team auf Probleme, und vor allem auf steile Lernkurven stieß.
Eine dieser Stellen war der Umgang mit Docker Containern.
Ein gewisses Grundverständnis war vorhanden, doch praktische Erfahrung war im gesamten Team nicht vorhanden.
Zwar was das Hochfahren einer Datenbank in einem Container schnell erreicht, doch ein Verständnis für Datenpersistenz und Volumes zu entwickeln, benötigte seine Zeit.
Auch im späteren Verlauf des Projekts, als es darum ging alle Applikationskomponenten in einer Docker-Compose Datei zusammenzuführen, stellte sich als Herausforderung heraus.
Wie sich Dockerfiles in dem ganzen System einordnen und wo sie genutzt werden, war ebenfalls neu für das Team.
Allgemein hat das Einfinden in Docker dem Team mehr Zeit als erwartet abverlangt.

Eine weitere Herausforderung war das Einfinden in die verschiedenen Bibliotheken, die bei Webentwicklung mit JavaScript genutzt werden.
Zwar wurde in der Planung bereits einige Bibliotheken festgelegt, die genutzt werden sollten, doch während der Entwicklung kamen einige neue hinzu.
Ein Beispiel hierfür ist die Bibliothek express-session, die für das Session-Management genutzt wird.
Da zur Planung noch kein detailliertes Verständnis darüber vorhanden war, welche Komponenten bei der Entwicklung einer solchen Anwendung benötigt werden, wurden Thematiken wie Session Management erst während der Entwicklung entdeckt.
Hierdurch kam es an einigen Stellen zu steilen, jedoch unvermeidbaren Lernkurven.

Im nächsten Schritt wurde die MongoDB implementiert.
Dies ist das erste Mal, dass das Team mit einer Dokumenten orientierten Datenbank arbeitet, daher musste erstmal ein Verständnis über den Aufbau einer solchen Datenbank geschaffen werden.
Zunächst wurde versucht vergleiche mit einer SQL Datenbank herzustellen, wobei schnell auffiel, dass Konzepte wie Schemata hier Collections sind.
Das integrieren der Datenbank in der Programmierumgebung war schnell erledigt, da dies analog zu der MySQL Datenbankverbindung erfolgt ist.
Das auslesen und schreiben von Daten in die Datenbank war dank der MongoDB Bibliothek einfach erledigt.
Funktionen für die Verbindung mit der Datenbank wurden in einer separaten Datei ausgelagert.
\subsection{Prototypvorstellung}\label{subsec:Prototyp}
Design Science: Artefakt
    \newpage


    \section{Ergebnis und Fazit}\label{sec:ergebnis-fazit}
    \input{sections/ergebnis-fazit}
    \newpage

    \appendix


    \section{Anhang}\label{sec:anhang}
    \subsection{Mockups}\label{subsec:Mockups}

\begin{figure}[htbp]
    \centering
    \includegraphics[width=0.6\textwidth]{add_collection_pop_up}
    \caption{'Add Collection' Pop-up}
\end{figure}

\begin{figure}[htbp]
    \centering
    \includegraphics[width=0.6\textwidth]{loggedIn_starting_site}
    \caption{'Logged in' page}
\end{figure}

\begin{figure}[htbp]
    \centering
    \includegraphics[width=0.6\textwidth]{Starting_site}
    \caption{Starting page}
\end{figure}


\end{document}