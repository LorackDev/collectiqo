\section{Reorganisation Projektstruktur}\label{sec:section-two}

Dieser Abschnitt befasst sich mit der Umstrukturierung des Projektverzeichnisses.

\subsection{Git-Repository}\label{subsec:subsection-two-one}

Das vorherige Semester war das erste mal, dass das Team eine Node JS Applikation entiwckelt hat.
Demnach wurde viel ausprobiert und es war noch kein Verständnis darüber vorhanden, wie eine solche Anwendung gut strukturiert aufgebaut wird.
Dieses Verständnis wurde über das letzte Semester aufgebautund es wurde beschlossen, das Projekt neu zu strukturieren.
Dies kostete zum Anfang des Semesters viel Zeit, hat jedoch den Effekt, dass eine einheitliche Struktur aufgebaut wurde und die Implementierung neuer Funktionen einfach sind und einem fest definierten Standard folgen.
Ein Prinzip, was aktiv verfolgt wurde, war das "Seperation of concerns" Prinzip.
Während zuvor alle bestandteile einer Funktion (Route, Response Codes und Funktionalität) in einer Datei gehandhabt wurden, wurden diese nun in drei seperate Dateien aufgeteilt.
Jede Route erhält eine eigene Datei, in welcher der Endpunkt für HTTP Request definiert wird und der entsprechende Controller aufgerufen wird.
Der Controller kümmert sich um den Aufruf der gewünschten Service Datei und handhabt die Response Codes, die beim Aufruf des Endpunkts zurück gegeben werden.
Die eigentliche Funktionalität ist in der Service Datei implementiert.
Diese Aufteilung bringt mehrere Vorteile mit sich.
Jede Funktionalität, die ab sofort implementiert wird, folgt genau diesem Muster, wodurch jede Implementierung für jedes Teammitglied klar nachverfolgbar ist.
Darüber hinaus erleichtert es das Debugging ungemein, da Probleme einfacher auf einzelne Bestandteile eines Funktionsaufrufs zurückgeführt werden können.
Kombiniert mit einer Ordnersturktur, die inhaltlich ferne Funktionen voneinander trennt, ist auf das navigieren innerhalb des Projets ereinfacht.
Schlusseindlich ist auch das Aufrufen vom HTTP Requests vereinfacht, da jeder Aufruf einheitlich ist.

\subsection{Docker Containerisierung}\label{subsec:subsection-two-two}

Während der Durchführung des Projekts ist ein Teammitglied auf einen ARM-Prozessor umgestiegen.
Letztes Semester erfolgte die Programmierung ausschließlich auf x86 Maschinen, wodruch Docker Images, die auf einer Maschine gebaut wurden, problemlos auf allen anderen Maschinen funktioniert haben.
Images, die für x86 Architektur gebaut wurden, funktionieren nicht auf ARM-Maschinen.
Eine Lösung wurde hier in Multiarchitektur Docker Images gefunden.
Wird ein solches Image aus einer Registry auf den eigenen Rechner gezogen, so erkennt Docker automatisch die Architektur, die benötigt wird.
So lässt sich bspw eine x86 und eine ARM Version des gleichen Docker Images in Docker Hub hosten.
Beim pullen des Images wird dann automatisch erkannt, für welche Architektur das Image benötigt wird.
Eine ARM Maschine zieht sich dann entsprechend auch nur das ARM Image.

Bevor die Images gehostet werden können, müssen diese aber zuerst gebaut werden.
Das Bauen von Multiarchitektur Images erfolgt mit einer Komponentne von Docker, die sich buildx nennt.
Buildx selbst ist hierbei eine Anwendung, die selbst in einem Docker Container läuft.
Damit interagiert wird über den üblichen docker build befehl,
Will man nun ein Image bauen, so übergibt man die Dateien an den buildx Container, in welchem dann Image erstellt wird.
Die gewünschten Architekturen übergibt man hier als Parameter.