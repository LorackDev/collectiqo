\section{Designanalyse}\label{sec:section-three}

Zu Beginn der Projektarbeit wurde eine umfassende Designanalyse durchgeführt und sorgfältig dokumentiert.
Im Rahmen dieser Analyse wurden verschiedene zentrale Bestandteile des bestehenden Designs betrachtet, darunter die Loginpage, die Landingpage, die Sidebar, Menüs, die Collectionseite einschließlich der Ansicht einzelner Kollektionen sowie die Accounteinstellungen.

Auffällig war insbesondere die Farbpalette des bisherigen Designs.
Obwohl ursprünglich mit einem festgelegten Farbschema gearbeitet wurde, konnten Abweichungen festgestellt werden.
Diese Abweichungen sind größtenteils auf Zeitmangel während der Umsetzung des vorherigen Projekts zurückzuführen.

Des Weiteren wurde festgestellt, dass für das bisherige Design generierte Bilder verwendet wurden.
Diese sollen unverändert erhalten bleiben, da sie gut zum Gesamtkonzept passen.
Zur Veranschaulichung der Startseite wurde außerdem ein Photoshop-Mockup erstellt, welches als erste Orientierung für die Überarbeitung des Designs dient.

Für jede wesentliche Seite des Projekts wurde anschließend eine gezielte Designrecherche betrieben.
Im Zuge dieser Recherche zeigte sich, dass bestimmte gestalterische Ähnlichkeiten zwischen den untersuchten Designs bestehen, die wir als Inspirationsquelle nutzen und bewusst in unser überarbeitetes Design integrieren möchten.

\subsection{Rechtliche Vorgaben}\label{subsec:subsection-three-one}

Bei der Gestaltung und dem Betrieb von Webseiten sind neben ästhetischen und funktionalen Aspekten auch verschiedene rechtliche Vorgaben zu beachten.
In diesem Abschnitt werden die wichtigsten rechtlichen Anforderungen für Webseiten zusammengefasst, die auch für Collectiqo eine Relevanz darstellen können.

\textbf{Datenschutz}\par
Einer der zentralen Vorgaben ist die Einhaltung der Datenschutz-Grundverordnung (DSGVO).
Webseiten müssen transparent über die Erhebung und Verarbeitung personenbezogener Daten informieren.
Das betrifft insbesondere die Verwendung von Cookies und die Speicherung von Nutzerdaten.
Beides trifft auch auf Collectiqo zu, da Nutzer Accounts anlegen können und damit persönliche Daten speichern.
Gleichwohl werden zur Aufrechterhaltung des Logins sogenannte Session-Cookies verwendet.
Der Server speichert für einen begrenzten Zeitraum zudem Logs, die unter anderem die IP-Adresse des Nutzers enthalten.

Ungeachtet ob es sich um eine gewerbliche oder private Webseite handelt, ist daher eine Disclaimer-Seite zwecks Datenschutz unerlässlich.
Diesem sind wir nachgekommen, indem eine gesonderte Seite, die jederzeit über den Header erreichbar ist, erstellt wurde.
Alle relevanten Informationen zur Datenerhebung und -verarbeitung sind dort aufgeführt.
Auch die Verwendung von Google Fonts wurde in der Datenschutzerklärung aufgeführt.

\textbf{Impressum}\par
Ein weiterer wichtiger rechtlicher Aspekt ist die Impressumspflicht.
Gewerbliche Webseiten, bzw. solche, die einen wirtschaftlichen Zweck verfolgen, müssen ein Impressum bereitstellen.
Dieses muss auf der Webseite leicht auffindbar sein und vollständige Angaben zum Verantwortlichen und Kontaktmöglichkeiten enthalten.
Hier war bis vor kurzem stets das Telemediengesetz (TMG) relevant, das unter anderem die Pflicht zur Bereitstellung eines Impressums regelt.
Seit dem Inkrafttreten des Digitale-Dienste-Gesetzes (DDG) am 14. Mai 2024 hat es das TMG abgelöst.

Auch wenn Collectiqo keine wirtschaftlichen Interessen verfolgt, ist ein Impressum dennoch sinnvoll, um die Transparenz und Glaubwürdigkeit der Seite zu erhöhen.
Dem Impressum sind wir also freiwillig nachgekommen, indem wir eine entsprechende Seite erstellt haben, die ebenfalls über den Header erreichbar ist.

\textbf{Barrierefreiheit}\par
Ein weiterer Aspekt ist die Barrierefreiheit.
Dazu gehört die Bereitstellung von alternativen Texten für Bilder, eine intuitive Navigation und eine klare, kontrastreiche Gestaltung.
Insbesondere öffentliche Institutionen sind durch die EU-Richtlinie 2016/2102 verpflichtet, ihre Webseiten barrierefrei zu gestalten.
Auch relevant ist das Barrierefreiheitsstärkungsgesetz (BFSG), 2021 verabschiedet, dass mit der Verordnung zum Barrierefreiheitsstärkungsgesetz (BFSGV) am 15.06.2022 verabschiedet zum 28. Juni 2025 in Kraft tritt.
Dieses Gesetz verpflichtet alle Unternehmen, dass digitale Inhalte für alle zugänglich gemacht werden, insbesondere für Menschen mit Behinderungen.

Auf Collectiqo trifft dies insofern nicht zu, da es sich um eine rein private Webseite handelt, die im Rahmen eines studentischen Projekts entstanden ist und keinen wirtschaftlichen und gewerbliche Zwecke verfolgt.

%\parencite{GIDF.2024}
% TODO: Just for Testing, you know! ;)

\subsection{Visuelles Redesign}\label{subsec:subsection-three-two}

Im Rahmen des Redesigns wurden zentrale Änderungen vorgenommen, um das bestehende Design zu modernisieren, benutzerfreundlicher zu gestalten und zu vereinheitlichen.
Eine der ersten Maßnahmen war die Anpassung der Farbpalette, die vollständig auf harmonische Lilatöne umgestellt wurde.
Dies erlaubt für eine potentielle Einführung eines Dunkel-Hell Modi und bietet eine bessere Ästhetik hinblickend auf die Philosophie von Collectiqo.

Alte Seiten, die zuvor den Designcode gebrochen hatten, wurden überarbeitet, um eine einheitliche visuelle Sprache sicherzustellen.
Zusätzlich wurde ein modernes, rundes Theme eingeführt, das die Benutzeroberfläche klarer und ansprechender macht.
Dies ist insbesondere beim Erstellen von Collections und der Accounteinstellungen einzusehen.

Ein wichtiger Schritt im Redesign war die Zusammenlegung und Komprimierung verschiedener Seiten.
Es wurde die Startseite mit dem Sign-Up verknüpft, und die Accounteinstellungen wurden in verschiedene Reiter gelegt.
Dadurch wurde die Benutzerführung vereinfacht und die Navigation für neue Nutzer intuitiver gestaltet.

Zur Verbesserung der Interaktion wurden Übergänge zwischen den Seiten implementiert, die das Nutzungserlebnis flüssiger gestalten sollen.
Allerdings lief die Umsetzung dieser Übergänge nicht immer optimal, was als zukünftiger Optimierungsbereich identifiziert wurde.