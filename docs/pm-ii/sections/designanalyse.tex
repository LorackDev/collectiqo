\section{Designanalyse}\label{sec:section-three}

Zu Beginn der Projektarbeit wurde eine umfassende Designanalyse durchgeführt und sorgfältig dokumentiert.
Im Rahmen dieser Analyse wurden verschiedene zentrale Bestandteile des bestehenden Designs betrachtet, darunter die Loginpage, die Landingpage, die Sidebar, Menüs, die Collectionseite einschließlich der Ansicht einzelner Kollektionen sowie die Accounteinstellungen.

Auffällig war insbesondere die Farbpalette des bisherigen Designs.
Obwohl ursprünglich mit einem festgelegten Farbschema gearbeitet wurde, konnten Abweichungen festgestellt werden.
Diese Abweichungen sind größtenteils auf Zeitmangel während der Umsetzung des vorherigen Projekts zurückzuführen.

Des Weiteren wurde festgestellt, dass für das bisherige Design generierte Bilder verwendet wurden.
Diese sollen unverändert erhalten bleiben, da sie gut zum Gesamtkonzept passen.
Zur Veranschaulichung der Startseite wurde außerdem ein Photoshop-Mockup erstellt, welches als erste Orientierung für die Überarbeitung des Designs dient.

Für jede wesentliche Seite des Projekts wurde anschließend eine gezielte Designrecherche betrieben.
Im Zuge dieser Recherche zeigte sich, dass bestimmte gestalterische Ähnlichkeiten zwischen den untersuchten Designs bestehen, die wir als Inspirationsquelle nutzen und bewusst in unser überarbeitetes Design integrieren möchten.

\subsection{Rechtliche Vorgaben}\label{subsec:subsection-three-one}

Bei der Gestaltung von Webseiten sind neben ästhetischen und funktionalen Aspekten auch rechtliche Vorgaben zu beachten.
Eine der zentralen Vorgaben ist die Einhaltung der Datenschutz-Grundverordnung (DSGVO).
Webseiten müssen transparent über die Erhebung und Verarbeitung personenbezogener Daten informieren.
Dazu gehören eine gut zugängliche Datenschutzerklärung sowie ein Cookie-Banner, das Nutzern die Möglichkeit gibt, der Nutzung von Cookies aktiv zuzustimmen oder sie abzulehnen.

Zusätzlich ist das Telemediengesetz (TMG) relevant, das unter anderem die Pflicht zur Bereitstellung eines Impressums regelt.
Dieses muss auf der Webseite leicht auffindbar sein und vollständige Angaben zu Verantwortlichen und Kontaktmöglichkeiten enthalten.

Ein weiterer wichtiger Aspekt ist die Barrierefreiheit, insbesondere für öffentliche Institutionen, die durch die EU-Richtlinie 2016/2102 verpflichtet sind, ihre Webseiten barrierefrei zu gestalten.
Dazu gehört die Bereitstellung von alternativen Texten für Bilder, eine intuitive Navigation und eine klare, kontrastreiche Gestaltung.
Dieses wurde aufgrund der Eingrenzung als bisher nicht-öffentliches Universitätsprojekt vorerst nicht beachtet.

\parencite{GIDF.2024}
% TODO: Just for Testing, you know! ;)

\subsection{Visuelles Redesign}\label{subsec:subsection-three-two}

Im Rahmen des Redesigns wurden zentrale Änderungen vorgenommen, um das bestehende Design zu modernisieren, benutzerfreundlicher zu gestalten und zu vereinheitlichen.
Eine der ersten Maßnahmen war die Anpassung der Farbpalette, die vollständig auf harmonische Lilatöne umgestellt wurde.
Dies erlaubt für eine potentielle Einführung eines Dunkel-Hell Modi und bietet eine bessere Ästhetik hinblickend auf die Philosophie von Collectiqo.

Alte Seiten, die zuvor den Designcode gebrochen hatten, wurden überarbeitet, um eine einheitliche visuelle Sprache sicherzustellen.
Zusätzlich wurde ein modernes, rundes Theme eingeführt, das die Benutzeroberfläche klarer und ansprechender macht.
Dies ist insbesondere beim Erstellen von Collections und der Accounteinstellungen einzusehen.

Ein wichtiger Schritt im Redesign war die Zusammenlegung und Komprimierung verschiedener Seiten.
Es wurde die Startseite mit dem Sign-Up verknüpft, und die Accounteinstellungen wurden in verschiedene Reiter gelegt.
Dadurch wurde die Benutzerführung vereinfacht und die Navigation für neue Nutzer intuitiver gestaltet.

Zur Verbesserung der Interaktion wurden Übergänge zwischen den Seiten implementiert, die das Nutzungserlebnis flüssiger gestalten sollen.
Allerdings lief die Umsetzung dieser Übergänge nicht immer optimal, was als zukünftiger Optimierungsbereich identifiziert wurde.