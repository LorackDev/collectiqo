\section{Projektplanung- und Management}\label{sec:section-one}

In diesem Kapitel wird in Abschnitt~\ref{subsec:subsection-one-one} zu Beginn der bisherige Stand aus dem ersten Praxismodul zusammengefasst.
Abschnitt~\ref{subsec:subsection-one-two} behandelt die Zielsetzung dieses zweiten Praxismoduls.
Die aktuelle Projektorganisation wird im dritten Abschnitt~\ref{subsec:subsection-one-three} definiert.
Abschnitt~\ref{subsec:subsection-one-four} gibt abschließend einen aktualisierten Überblick über die verwendeten Tools.

\subsection{Rückblick auf Praxismodul I}\label{subsec:subsection-one-one}

Das Praxismodul I begann mit der Projektidee und Motivation eine webbasierte Plattform, \textbf{\textit{Collectiqo}} genannt, zu erstellen um Sammlungen \textit{(eng. collections)} aller Art übersichtlich zu katalogisieren.
Innerhalb der Projektdokumentation umfassten zwei umfangreichere Kapitel die Projektplanung und das Projektmanagement.
Die Planung bestand aus den Abschnitten Zieldefinition, Teilergebnisse, Projektsteckbrief, Definitions of Done und Projektstrukturplan.
Das Projektmanagement behandelte die Abschnitte Organigramm, Ablaufplanung, Stakeholder- und Risikoanalyse, Kosten- und Aufwandsplanung, sowie verwendete Tools.

Die Projektstruktur basierte auf einem Git Repository, Administration und Programmierung erfolgte mithilfe der IDE von Webstorm.
Anhand Docker wurde eine dreigeteilte Containerisierung genutzt.
Container 1 stelle das Front- und Backend dar, die anderen beiden Container waren jeweils die Datenbanksysteme MySQL (User-Management) und mongoDB (Sammlungen).
Die Programmiersprache war JavaScript, das Frontend wurde mit HTML, CSS und EJS entwickelt, das Backend setzt auf Node.js, express und axios.

Zum Ende des Praxismoduls I wurde die Plattform in einer Präsentation vorgestellt und die Projektdokumentation abgegeben.
Viele grundlegende Ziele wurden erfolgreich abgeschlossen, darunter die grundlegende Implementierung Backend- und Frontend-Systeme.
Das Datenbanksystem wurde vollständig umgesetzt.
Die vollumfängliche Verknüpfung der Systeme war noch in Bearbeitung, insbesondere die Verbindung zwischen Frontend und Backend.
Account-Erstellung, Benutzer-Login und die Nutzung vorhandener Sammlungs-Templates funktionierten bereits.
Das Anlegen eigener Templates war im Backend implementiert, fehlte aber noch im Frontend.
Einfache Unit-Tests wurden erfolgreich implementiert.

Viele der geplanten Ziele wurden erreicht, dennoch traten einige Herausforderungen und Verzögerungen auf.
Das Team erwarb umfangreiche neue Kenntnisse in verschiedenen Bereichen, die Lernkurve war allerdings steiler als zuvor erwartet, was auf die Komplexität der Webentwicklung und neue Tools wie Docker bzw. das dokumentenorientierte NoSQL-Datenbankmanagementsystem MongoDB zurückzuführen war.
Vereinzelte Ziele mussten dadurch allerdings zeitlich verschoben werden oder waren zum Zeitpunkt der Abgabe des Praxismoduls I noch in Bearbeitung.

%TODO: ggf. noch 3-5 Sätze hinzufügen um den Seitenplatz voll auszunutzen
\textbf{Hier ist noch Platz für einen weiteren Satz.}
\textbf{Hier ist noch Platz für einen weiteren Satz.}
\textbf{Hier ist noch Platz für einen weiteren Satz.}
\textbf{Hier ist noch Platz für einen weiteren Satz.}
\textbf{Hier ist noch Platz für einen weiteren Satz.}
\textbf{Hier ist noch Platz für einen weiteren Satz.}
\newpage

\subsection{Zielsetzungen im Praxismodul II}\label{subsec:subsection-one-two}

Das Praxismodul II baut auf den Ergebnissen des ersten Praxismoduls auf und setzt die Entwicklung der Plattform \textbf{\textit{Collectiqo}} fort.
Dies umfasst sowohl die Fertigstellung geplanter, aber noch nicht umgesetzter Funktionen als auch die Implementierung neuer Ideen.
Die Projektplanung und -organisation soll dabei deutlich kürzer gehalten werden, da diese bereits im ersten Praxismodul ausführlich behandelt wurde.
Hauptaugenmerk liegt daher auf der technischen Implementierung und der Erweiterung der Funktionalitäten.

%TODO: Richtige bzw. logische Reihenfolge der nachfolgenden Punkte/Themen mit leicht begelitendem Prosa-Text

Ein Hauptfokus liegt auf der Erweiterung der Sammlungsverwaltung, insbesondere der Erstellung eigener Templates und dem Editieren bestehender Sammlungen.

Im Frontend ist zudem ein visuelles Redesign geplant, basierend auf den bereits erstellten HTML-Grundlagen.

Das Backend soll robuster gestaltet und die Verknüpfung mit dem Frontend vervollständigt werden.

Auf Datenbankebene ist die Absicherung durch Constraints vorgesehen, um die Datenintegrität zu gewährleisten.

Kleinere Bugs in Front- und Backend sollen behoben werden.

Die Dockerumgebung wird erweitert, indem Client- und Serverseite in separate Container aufgeteilt werden.

Ein wichtiger Schwerpunkt liegt auf der Erstellung von Integrationstests, die die bestehenden einfachen Unit-Tests ergänzen sollen.

Das Projektmanagementtool YouTrack soll durch ein übersichtlicheres und in der Industrie verbreiteteres Tool ersetzt werden.

Eine Umstrukturierung des Git-Repositories ist geplant, um die Übersichtlichkeit zu verbessern und den Entwicklungsworkflow zu optimieren.

% TODO: Sofern die Sharing Funktion verfolgt wird!
Als neue Funktion ist die Implementierung einer Sharing-Möglichkeit für Sammlungen vorgesehen.
Dieser Community-Aspekt erfordert auch einen Ausbau der Benutzereinstellungen.


\subsection{Projektorganisationen}\label{subsec:subsection-one-three}
% TODO: Ausformulierung noch ausstehend

\begin{itemize}[noitemsep]
    \item Betreuer dankenswerterweise weiterhin Herr Prof. Dr. Dirk Schweim
    \item Kleine Änderungen Organigramm verschriftlichen.
    \item Keine detaillierte Zeitplanung bzw. PSP, da bereits im ersten Praxismodul ausführlich behandelt.
    \item Jira als Tool zur Projektorganisation und -verwaltung.
\end{itemize}

%TODO: Organigrammfunktionen und Rollenverteilung richtig darstellen
Anika: Grafische Frontendentwicklung basierend auf HTML, CSS und EJS
Darko: Technische Frontendentwicklung JavaScript
Lorenz: Backend-Entwicklung mit Node.js, express und axios; Integrationstests
Robin: Datenbanksysteme und Docker


\subsection{Tools}\label{subsec:subsection-one-four}
% TODO: Ausformulierung noch ausstehend

Änderungen / Delta zu PM I erläutern:
\begin{itemize}[noitemsep]
    \item Jira
    \item Photoshop zu vergünstigten Bildungskonditionen
    \item Placeholder aktuelles Bild "Übersicht Tools"
\end{itemize}


