\section{Fazit und Ausblick}\label{sec:fazit-ausblick}

Dieses Kapitel fasst die wichtigsten Erkenntnisse zusammen, beleuchtet Abweichungen von den ursprünglichen Zielen und reflektiert die Fortschritte des Projekts.
Gleichzeitig wird ein Ausblick auf zukünftige Schritte gegeben, um das Projekt weiterzuentwickeln und an neue Herausforderungen anzupassen.

\subsection{Abweichungen der geplanten Ziele}\label{subsec:abweichungen-der-geplanten-ziele}

Im Rahmen des Projekts konnten einige der gesetzten Ziele erfolgreich umgesetzt werden, während andere aufgrund unvorhergesehener Herausforderungen nicht erreicht wurden.

\begin{itemize}[noitemsep]
    \item \textbf{Restrukturierung der GitHub-Struktur}: Dieses Ziel wurde vollständig erreicht, wodurch eine klarere Organisation und Nachvollziehbarkeit des Projekts gewährleistet ist.
    \item \textbf{Auslagerung von redundanten Code-Snippets in HTML, CSS und JavaScript}: Die Reduktion von Redundanzen wurde erfolgreich umgesetzt, was die Wartbarkeit und Skalierbarkeit des Projekts deutlich verbessert hat.
    \item \textbf{Designanpassung mit richtigem Branding}: Das Branding wurde erfolgreich implementiert, was zu einem professionelleren und kohärenten Auftritt beiträgt.
    \item \textbf{Projektmanagement-Tool}: Der Wechsel auf ein neues Projektmanagement-Tool konnte zu Beginn des Praxismoduls II erfolgreich umgesetzt werden, was die Organisation und Kommunikation im Team verbessert hat.
    \item \textbf{Docker-Erweiterung}: Die Docker-Umgebung konnte erfolgreich um zwei Container (Reverse Proxy und Certbot Zertifikatsverwaltung) erweitert werden.
    \item \textbf{Bugs im Front- und Backend}: Die Behebung von Bugs wurde erfolgreich durchgeführt, was die Stabilität und Zuverlässigkeit der Plattform erhöht hat.
    \item \textbf{vServer-Hosting}: Die Plattform konnte erfolgreich auf einem vServer gehostet werden, was die Verfügbarkeit verbessert und die Grundlage für zukünftige Erweiterungen schafft.
    \item \textbf{DB-Constrains}: Die Absicherung der Datenbank durch Constraints konnte aus Zeitgründen nicht umgesetzt werden, bleibt aber ein Ziel für zukünftige Entwicklungen.
    \item \textbf{Integrationstests}: Die Implementierung von Integrationstests konnte ebenfalls nicht umgesetzt werden und verbleibt im Backlog.
    \item \textbf{Erweiterung der Funktionen für die Anpassung von Collections}: Dieses Ziel konnte nicht erreicht werden, da die notwendigen Umstrukturierungen und die Vereinheitlichung anderer Komponenten mehr Zeit in Anspruch nahmen als ursprünglich geplant.
\end{itemize}

\newpage

\subsection{Retrospektive}\label{subsec:retrospektive}

Im Vergleich zum ersten Semester haben wir uns bewusst weniger vorgenommen, um einen klaren Fokus auf die grundlegende Restrukturierung des Projektes zu legen.
Unser Hauptziel war es, die Wiederverwendbarkeit von Code zu ermöglichen und damit eine nachhaltige Grundlage für zukünftige Entwicklungen zu schaffen.
Die Teammitglieder konnten zudem gezielter persönlichen Interessen nachgehen und sich in spezifischen Bereichen vertiefen.

Ein weiterer Schwerpunkt lag auf der Verbesserung der Benutzerfreundlichkeit und der Gestaltung ansprechender Designs.
Diese Maßnahmen sollten sicherstellen, dass spätere Funktionserweiterungen ohne größere Rückschritte oder grundlegende Änderungen umgesetzt werden können.

Durch diese gezielte Herangehensweise haben wir eine solide Basis geschaffen, die sowohl die technische als auch die visuelle Qualität des Projekts langfristig unterstützt.

\subsection{Ausblick}\label{subsec:ausblick-zukuenftige-ziele-und-funktionen}

Das Projektteam strebt an, die Plattform \textbf{Collectiqo} auch im Praxismodul III weiterzuentwickeln und zu verbessern.
Zukünftig sind weitere Erweiterungen und Verbesserungen geplant, um die Funktionalität bzw. Sicherheit der Plattform zu steigern und auszubauen.
Dazu gehören nach aktuellem Stand überlegungen in Richtung:

\begin{itemize}[noitemsep]
    \item Vollständige Collectionansicht mit Bearbeitung der Collections.
    \item Änderbare Einstellungen der Benutzerdaten durch User selbst.
    \item Ausbau der vServer-Infrastruktur, Optimierung und Fortentwicklung von Funktionalitäten.
    \item Betrachtung der Anmeldung einer Wortmarke für rechtlichen Schutz.
    \item Implementierung eines E-Mail-Bestätigungscodes für die Registrierung sowie eine „Passwort vergessen“-Funktion.
    \item Verteiltes Arbeiten an der Plattform in dedizierten branches.
    \item Pipelines für Auto-Deployment auf dem vServer.
\end{itemize}


