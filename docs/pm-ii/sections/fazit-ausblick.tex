\section{Fazit und Ausblick}\label{sec:fazit-ausblick}

Dieses Kapitel fasst die wichtigsten Erkenntnisse zusammen, beleuchtet Abweichungen von den ursprünglichen Zielen und reflektiert die Fortschritte des Projekts.
Gleichzeitig wird ein Ausblick auf zukünftige Schritte gegeben, um das Projekt weiterzuentwickeln und an neue Herausforderungen anzupassen.

\subsection{Abweichungen der geplanten Ziele}\label{subsec:abweichungen-der-geplanten-ziele}

Im Rahmen des Projekts konnten einige der gesetzten Ziele erfolgreich umgesetzt werden, während andere aufgrund unvorhergesehener Herausforderungen nicht erreicht wurden.

\begin{itemize}[noitemsep]
    \item \textbf{Restrukturierung der GitHub-Struktur}: Dieses Ziel wurde vollständig erreicht, wodurch eine klarere Organisation und Nachvollziehbarkeit des Projekts gewährleistet ist.
    \item \textbf{Auslagerung von redundanten Code-Snippets in HTML, CSS und JavaScript}: Die Reduktion von Redundanzen wurde erfolgreich umgesetzt, was die Wartbarkeit und Skalierbarkeit des Projekts deutlich verbessert hat.
    \item \textbf{Designanpassung mit richtigem Branding}: Das Branding wurde erfolgreich implementiert, was zu einem professionelleren und kohärenten Auftritt beiträgt.
    \item \textbf{Erweiterung der Funktionen für die Anpassung von Collections}: Dieses Ziel konnte nicht erreicht werden, da die notwendigen Umstrukturierungen und die Vereinheitlichung anderer Komponenten mehr Zeit in Anspruch nahmen als ursprünglich geplant.
\end{itemize}

\subsection{Retrospektive}\label{subsec:retrospektive}

Im Vergleich zum ersten Semester haben wir uns bewusst weniger vorgenommen, um einen klaren Fokus auf die grundlegende Restrukturierung des Projektes zu legen.
Unser Hauptziel war es, die Wiederverwendbarkeit von Code zu ermöglichen und damit eine nachhaltige Grundlage für zukünftige Entwicklungen zu schaffen.

Ein weiterer Schwerpunkt lag auf der Verbesserung der Benutzerfreundlichkeit und der Gestaltung ansprechender Designs.
Diese Maßnahmen sollten sicherstellen, dass spätere Funktionserweiterungen ohne größere Rückschritte oder grundlegende Änderungen umgesetzt werden können.

Durch diese gezielte Herangehensweise haben wir eine solide Basis geschaffen, die sowohl die technische als auch die visuelle Qualität des Projekts langfristig unterstützt.

\subsection{Ausblick}\label{subsec:ausblick-zukuenftige-ziele-und-funktionen}

Zukünftig sind weitere Erweiterungen und Verbesserungen geplant, um die Funktionalität und Sicherheit der Plattform zu steigern.
Dazu gehören unter anderem:

\begin{itemize}[noitemsep]
    \item Vollständige Colletionansicht mit bearbeiten der Collection.
    \item Änderbare Einstellungen der Benutzerdaten durch User selbst.
    \item Einrichtung eines Live Hostings, beispielsweise auf einem Strato-vServer, mit Zugangsbeschränkung auf das Netzwerk der HS Mainz.
    \item Registrierung einer eigenen Domain zur Professionalisierung der Plattform.
    \item Betrachtung der Anmeldung einer Wortmarke für rechtlichen Schutz.
    \item Implementierung eines E-Mail-Bestätigungscodes für die Registrierung sowie eine „Passwort vergessen“-Funktion.
\end{itemize}


