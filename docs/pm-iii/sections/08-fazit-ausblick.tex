\section{Fazit und Ausblick}\label{sec:fazit-ausblick}

Lorenz ipsum dolor sit amet, consetetur sadipscing elitr, sed diam nonumy eirmod tempor invidunt ut labore et dolore magna aliquyam erat, sed diam voluptua.
At vero eos et accusam et justo duo dolores et ea rebum.
Stet clita kasd gubergren, no sea takimata sanctus est Lorenz ipsum dolor sit amet.

\subsection{Abweichungen der geplanten Ziele}\label{subsec:abweichungen-der-geplanten-ziele}

Lorenz ipsum dolor sit amet, consetetur sadipscing elitr, sed diam nonumy eirmod tempor invidunt ut labore et dolore magna aliquyam erat, sed diam voluptua.
At vero eos et accusam et justo duo dolores et ea rebum.
Stet clita kasd gubergren, no sea takimata sanctus est Lorenz ipsum dolor sit amet.

\begin{itemize}[noitemsep]
    \item \textbf{Restrukturierung der GitHub-Struktur}: Dieses Ziel wurde vollständig erreicht, wodurch eine klarere Organisation und Nachvollziehbarkeit des Projekts gewährleistet ist.
    \item \textbf{Auslagerung von redundanten Code-Snippets in HTML, CSS und JavaScript}: Die Reduktion von Redundanzen wurde erfolgreich umgesetzt, was die Wartbarkeit und Skalierbarkeit des Projekts deutlich verbessert hat.
    \item \textbf{Designanpassung mit richtigem Branding}: Das Branding wurde erfolgreich implementiert, was zu einem professionelleren und kohärenten Auftritt beiträgt.
    \item \textbf{Projektmanagement-Tool}: Der Wechsel auf ein neues Projektmanagement-Tool konnte zu Beginn des Praxismoduls II erfolgreich umgesetzt werden, was die Organisation und Kommunikation im Team verbessert hat.
    \item \textbf{Docker-Erweiterung}: Die Docker-Umgebung konnte erfolgreich um zwei Container (Reverse Proxy und Certbot Zertifikatsverwaltung) erweitert werden.
    \item \textbf{Bugs im Front- und Backend}: Die Behebung von Bugs wurde erfolgreich durchgeführt, was die Stabilität und Zuverlässigkeit der Plattform erhöht hat.
    \item \textbf{vServer-Hosting}: Die Plattform konnte erfolgreich auf einem vServer gehostet werden, was die Verfügbarkeit verbessert und die Grundlage für zukünftige Erweiterungen schafft.
    \item \textbf{DB-Constrains}: Die Absicherung der Datenbank durch Constraints konnte aus Zeitgründen nicht umgesetzt werden, bleibt aber ein Ziel für zukünftige Entwicklungen.
    \item \textbf{Integrationstests}: Die Implementierung von Integrationstests konnte ebenfalls nicht umgesetzt werden und verbleibt im Backlog.
    \item \textbf{Erweiterung der Funktionen für die Anpassung von Collections}: Dieses Ziel konnte nicht erreicht werden, da die notwendigen Umstrukturierungen und die Vereinheitlichung anderer Komponenten mehr Zeit in Anspruch nahmen als ursprünglich geplant.
\end{itemize}

\newpage

\subsection{Retrospektive}\label{subsec:retrospektive}

Lorenz ipsum dolor sit amet, consetetur sadipscing elitr, sed diam nonumy eirmod tempor invidunt ut labore et dolore magna aliquyam erat, sed diam voluptua.
At vero eos et accusam et justo duo dolores et ea rebum.
Stet clita kasd gubergren, no sea takimata sanctus est \textbf{Lorenz ipsum} dolor sit amet.

\subsection{Ausblick}\label{subsec:ausblick-zukuenftige-ziele-und-funktionen}

Lorenz ipsum dolor sit amet, consetetur sadipscing elitr, sed diam nonumy eirmod tempor invidunt ut labore et dolore magna aliquyam erat, sed diam voluptua.
At vero eos et accusam et justo duo dolores et ea rebum.
Stet clita kasd gubergren, no sea takimata sanctus est Lorenz ipsum dolor sit amet:

\begin{itemize}[noitemsep]
    \item Vollständige Collectionansicht mit Bearbeitung der Collections.
    \item Änderbare Einstellungen der Benutzerdaten durch User selbst.
    \item Ausbau der vServer-Infrastruktur, Optimierung und Fortentwicklung von Funktionalitäten.
    \item Betrachtung der Anmeldung einer Wortmarke für rechtlichen Schutz.
    \item Implementierung eines E-Mail-Bestätigungscodes für die Registrierung sowie eine „Passwort vergessen“-Funktion.
    \item Verteiltes Arbeiten an der Plattform in dedizierten branches.
    \item Pipelines für Auto-Deployment auf dem vServer.
\end{itemize}


