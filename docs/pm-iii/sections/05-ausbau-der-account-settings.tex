\section{Ausbau der Nutzereinstellungen}\label{sec:ausbau-der-nutzereinstellungen}

In diesem Kapitel wird die Erweiterung der Nutzereinstellungen beschrieben.
Ziel war es, Nutzern mehr Kontrolle über ihr Konto zu geben, insbesondere im Hinblick auf die Verwaltung von Zugangsdaten und Authentifizierungsinformationen.
Aufbauend auf dem bestehenden Frontend wurde das Backend um verschiedene Funktionalitäten ergänzt, welche die Selbstverwaltung der Nutzerkonten ermöglichen.

\subsection{Frontend}\label{subsec:frontend}

\subsection{Backend}\label{subsec:backend}

Im Rahmen des dritten Praxismoduls wurde das Backend zur Verwaltung von Nutzereinstellungen weiterentwickelt.
Aufbauend auf dem Frontend aus dem vorherigen Semester wurde die serverseitige Logik ergänzt, um Änderungen an Benutzerdaten sicher zu verarbeiten.
Die umgesetzten Funktionen umfassen die Änderung von Benutzername, Passwort und E-Mail-Adresse sowie das Ausloggen eines Nutzers.
Zur Datenverarbeitung wird eine Kombination aus MySQL (für Nutzerdaten) und MongoDB (für sammlungsbezogene Informationen) verwendet.
Die Implementierung folgt dem etablierten Aufbau aus Route, Controller und Service.
Da die Aufteilung bereits in der Dokumentation des Praxismoduls II erklärt wurde, wird hier nicht näher darauf eingegangen.

\subsubsection{Ändern des Benutzernamens}\label{subsubsec:username-update}

Die Funktion zur Änderung des Benutzernamens prüft zunächst, ob der alte Benutzername existiert und ob der neue Benutzername bereits vergeben ist.
Anschließend wird der Name sowohl in der MySQL-Datenbank als auch in den relevanten MongoDB-Collections aktualisiert:

\begin{lstlisting}[language=JavaScript, caption=Überprüfung und Update des Usernamens in MySQL]
const results = await queryDatabase('SELECT * FROM clq_users WHERE username = ?', [oldUsername]);
const user = handleResults(results);

if (!user) {
    throw new Error('User not found');
}

const results = await queryDatabase('SELECT * FROM clq_users WHERE username = ?', [newUsername]);
if (handleResults(results)) {
    throw new Error('Username already in use');
}

await queryDatabase('UPDATE clq_users SET username = ? WHERE username = ?', [newUsername, oldUsername]);
\end{lstlisting}

Zusätzlich wird der Benutzername in MongoDB aktualisiert, um Konsistenz über alle Datenbanken hinweg sicherzustellen.

\subsubsection{Ändern des Passworts}\label{subsubsec:password-update}

Zur Änderung des Passworts wird zunächst das bisherige Passwort mittels `bcrypt.compare` validiert.
Anschließend wird das neue Passwort gehasht und in der Datenbank gespeichert:

\begin{lstlisting}[language=JavaScript, caption=Passwortvalidierung und Update]
const passwordMatches = await bcrypt.compare(oldPassword, user.password);
if (!passwordMatches) {
    throw new Error('Incorrect old password');
}

const hashedNewPassword = await bcrypt.hash(newPassword, 10);
await queryDatabase('UPDATE clq_users SET password = ? WHERE username = ?', [hashedNewPassword, username]);
\end{lstlisting}

Diese Sicherheitsmaßnahmen stellen sicher, dass nur authentifizierte Nutzer ihr Passwort ändern können.

\subsubsection{Ändern der E-Mail-Adresse}\label{subsubsec:email-update}

Die Änderung der E-Mail-Adresse erfolgt durch eine einfache Aktualisierung des entsprechenden Feldes in der MySQL-Datenbank.
Vorab wird überprüft, ob der Benutzername existiert:

\begin{lstlisting}[language=JavaScript, caption=Update der E-Mail-Adresse]
const results = await queryDatabase('SELECT * FROM clq_users WHERE username = ?', [username]);
const user = handleResults(results);

if (!user) {
    throw new Error('User not found');
}

await queryDatabase('UPDATE clq_users SET email = ? WHERE username = ?', [newEmail, username]);
\end{lstlisting}

\subsubsection{Logout-Funktionalität}\label{subsubsec:logout}

Der Logout-Prozess basiert auf der Zerstörung der Session des Nutzers.
Dies wurde durch die `req.session.destroy()` Methode realisiert.
Bei erfolgreicher Durchführung wird die Session beendet, andernfalls ein Fehler zurückgegeben:

\begin{lstlisting}[language=JavaScript, caption=Logout-Service über Session-Zerstörung]
req.session.destroy(err => {
    if (err) {
        return reject(new Error('Failed to logout'));
    }
    resolve();
});
\end{lstlisting}